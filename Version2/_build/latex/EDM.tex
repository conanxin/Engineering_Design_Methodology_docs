% Generated by Sphinx.
\def\sphinxdocclass{report}
\documentclass[letterpaper,10pt,english]{sphinxmanual}
\usepackage[utf8]{inputenc}
\DeclareUnicodeCharacter{00A0}{\nobreakspace}
\usepackage{cmap}
\usepackage[T1]{fontenc}
\usepackage{babel}
\usepackage{times}
\usepackage[Bjarne]{fncychap}
\usepackage{longtable}
\usepackage{sphinx}
\usepackage{multirow}

\hypersetup{unicode=true}
\usepackage{CJKutf8}
\AtBeginDocument{\begin{CJK}{UTF8}{gbsn}}
\AtEndDocument{\end{CJK}}


\title{Engineering Design Methodology}
\date{December 24, 2013}
\release{1.0(beta)}
\author{ConanXin}
\newcommand{\sphinxlogo}{}
\renewcommand{\releasename}{Release}
\makeindex

\makeatletter
\def\PYG@reset{\let\PYG@it=\relax \let\PYG@bf=\relax%
    \let\PYG@ul=\relax \let\PYG@tc=\relax%
    \let\PYG@bc=\relax \let\PYG@ff=\relax}
\def\PYG@tok#1{\csname PYG@tok@#1\endcsname}
\def\PYG@toks#1+{\ifx\relax#1\empty\else%
    \PYG@tok{#1}\expandafter\PYG@toks\fi}
\def\PYG@do#1{\PYG@bc{\PYG@tc{\PYG@ul{%
    \PYG@it{\PYG@bf{\PYG@ff{#1}}}}}}}
\def\PYG#1#2{\PYG@reset\PYG@toks#1+\relax+\PYG@do{#2}}

\expandafter\def\csname PYG@tok@gd\endcsname{\def\PYG@tc##1{\textcolor[rgb]{0.63,0.00,0.00}{##1}}}
\expandafter\def\csname PYG@tok@gu\endcsname{\let\PYG@bf=\textbf\def\PYG@tc##1{\textcolor[rgb]{0.50,0.00,0.50}{##1}}}
\expandafter\def\csname PYG@tok@gt\endcsname{\def\PYG@tc##1{\textcolor[rgb]{0.00,0.27,0.87}{##1}}}
\expandafter\def\csname PYG@tok@gs\endcsname{\let\PYG@bf=\textbf}
\expandafter\def\csname PYG@tok@gr\endcsname{\def\PYG@tc##1{\textcolor[rgb]{1.00,0.00,0.00}{##1}}}
\expandafter\def\csname PYG@tok@cm\endcsname{\let\PYG@it=\textit\def\PYG@tc##1{\textcolor[rgb]{0.25,0.50,0.56}{##1}}}
\expandafter\def\csname PYG@tok@vg\endcsname{\def\PYG@tc##1{\textcolor[rgb]{0.73,0.38,0.84}{##1}}}
\expandafter\def\csname PYG@tok@m\endcsname{\def\PYG@tc##1{\textcolor[rgb]{0.13,0.50,0.31}{##1}}}
\expandafter\def\csname PYG@tok@mh\endcsname{\def\PYG@tc##1{\textcolor[rgb]{0.13,0.50,0.31}{##1}}}
\expandafter\def\csname PYG@tok@cs\endcsname{\def\PYG@tc##1{\textcolor[rgb]{0.25,0.50,0.56}{##1}}\def\PYG@bc##1{\setlength{\fboxsep}{0pt}\colorbox[rgb]{1.00,0.94,0.94}{\strut ##1}}}
\expandafter\def\csname PYG@tok@ge\endcsname{\let\PYG@it=\textit}
\expandafter\def\csname PYG@tok@vc\endcsname{\def\PYG@tc##1{\textcolor[rgb]{0.73,0.38,0.84}{##1}}}
\expandafter\def\csname PYG@tok@il\endcsname{\def\PYG@tc##1{\textcolor[rgb]{0.13,0.50,0.31}{##1}}}
\expandafter\def\csname PYG@tok@go\endcsname{\def\PYG@tc##1{\textcolor[rgb]{0.20,0.20,0.20}{##1}}}
\expandafter\def\csname PYG@tok@cp\endcsname{\def\PYG@tc##1{\textcolor[rgb]{0.00,0.44,0.13}{##1}}}
\expandafter\def\csname PYG@tok@gi\endcsname{\def\PYG@tc##1{\textcolor[rgb]{0.00,0.63,0.00}{##1}}}
\expandafter\def\csname PYG@tok@gh\endcsname{\let\PYG@bf=\textbf\def\PYG@tc##1{\textcolor[rgb]{0.00,0.00,0.50}{##1}}}
\expandafter\def\csname PYG@tok@ni\endcsname{\let\PYG@bf=\textbf\def\PYG@tc##1{\textcolor[rgb]{0.84,0.33,0.22}{##1}}}
\expandafter\def\csname PYG@tok@nl\endcsname{\let\PYG@bf=\textbf\def\PYG@tc##1{\textcolor[rgb]{0.00,0.13,0.44}{##1}}}
\expandafter\def\csname PYG@tok@nn\endcsname{\let\PYG@bf=\textbf\def\PYG@tc##1{\textcolor[rgb]{0.05,0.52,0.71}{##1}}}
\expandafter\def\csname PYG@tok@no\endcsname{\def\PYG@tc##1{\textcolor[rgb]{0.38,0.68,0.84}{##1}}}
\expandafter\def\csname PYG@tok@na\endcsname{\def\PYG@tc##1{\textcolor[rgb]{0.25,0.44,0.63}{##1}}}
\expandafter\def\csname PYG@tok@nb\endcsname{\def\PYG@tc##1{\textcolor[rgb]{0.00,0.44,0.13}{##1}}}
\expandafter\def\csname PYG@tok@nc\endcsname{\let\PYG@bf=\textbf\def\PYG@tc##1{\textcolor[rgb]{0.05,0.52,0.71}{##1}}}
\expandafter\def\csname PYG@tok@nd\endcsname{\let\PYG@bf=\textbf\def\PYG@tc##1{\textcolor[rgb]{0.33,0.33,0.33}{##1}}}
\expandafter\def\csname PYG@tok@ne\endcsname{\def\PYG@tc##1{\textcolor[rgb]{0.00,0.44,0.13}{##1}}}
\expandafter\def\csname PYG@tok@nf\endcsname{\def\PYG@tc##1{\textcolor[rgb]{0.02,0.16,0.49}{##1}}}
\expandafter\def\csname PYG@tok@si\endcsname{\let\PYG@it=\textit\def\PYG@tc##1{\textcolor[rgb]{0.44,0.63,0.82}{##1}}}
\expandafter\def\csname PYG@tok@s2\endcsname{\def\PYG@tc##1{\textcolor[rgb]{0.25,0.44,0.63}{##1}}}
\expandafter\def\csname PYG@tok@vi\endcsname{\def\PYG@tc##1{\textcolor[rgb]{0.73,0.38,0.84}{##1}}}
\expandafter\def\csname PYG@tok@nt\endcsname{\let\PYG@bf=\textbf\def\PYG@tc##1{\textcolor[rgb]{0.02,0.16,0.45}{##1}}}
\expandafter\def\csname PYG@tok@nv\endcsname{\def\PYG@tc##1{\textcolor[rgb]{0.73,0.38,0.84}{##1}}}
\expandafter\def\csname PYG@tok@s1\endcsname{\def\PYG@tc##1{\textcolor[rgb]{0.25,0.44,0.63}{##1}}}
\expandafter\def\csname PYG@tok@gp\endcsname{\let\PYG@bf=\textbf\def\PYG@tc##1{\textcolor[rgb]{0.78,0.36,0.04}{##1}}}
\expandafter\def\csname PYG@tok@sh\endcsname{\def\PYG@tc##1{\textcolor[rgb]{0.25,0.44,0.63}{##1}}}
\expandafter\def\csname PYG@tok@ow\endcsname{\let\PYG@bf=\textbf\def\PYG@tc##1{\textcolor[rgb]{0.00,0.44,0.13}{##1}}}
\expandafter\def\csname PYG@tok@sx\endcsname{\def\PYG@tc##1{\textcolor[rgb]{0.78,0.36,0.04}{##1}}}
\expandafter\def\csname PYG@tok@bp\endcsname{\def\PYG@tc##1{\textcolor[rgb]{0.00,0.44,0.13}{##1}}}
\expandafter\def\csname PYG@tok@c1\endcsname{\let\PYG@it=\textit\def\PYG@tc##1{\textcolor[rgb]{0.25,0.50,0.56}{##1}}}
\expandafter\def\csname PYG@tok@kc\endcsname{\let\PYG@bf=\textbf\def\PYG@tc##1{\textcolor[rgb]{0.00,0.44,0.13}{##1}}}
\expandafter\def\csname PYG@tok@c\endcsname{\let\PYG@it=\textit\def\PYG@tc##1{\textcolor[rgb]{0.25,0.50,0.56}{##1}}}
\expandafter\def\csname PYG@tok@mf\endcsname{\def\PYG@tc##1{\textcolor[rgb]{0.13,0.50,0.31}{##1}}}
\expandafter\def\csname PYG@tok@err\endcsname{\def\PYG@bc##1{\setlength{\fboxsep}{0pt}\fcolorbox[rgb]{1.00,0.00,0.00}{1,1,1}{\strut ##1}}}
\expandafter\def\csname PYG@tok@kd\endcsname{\let\PYG@bf=\textbf\def\PYG@tc##1{\textcolor[rgb]{0.00,0.44,0.13}{##1}}}
\expandafter\def\csname PYG@tok@ss\endcsname{\def\PYG@tc##1{\textcolor[rgb]{0.32,0.47,0.09}{##1}}}
\expandafter\def\csname PYG@tok@sr\endcsname{\def\PYG@tc##1{\textcolor[rgb]{0.14,0.33,0.53}{##1}}}
\expandafter\def\csname PYG@tok@mo\endcsname{\def\PYG@tc##1{\textcolor[rgb]{0.13,0.50,0.31}{##1}}}
\expandafter\def\csname PYG@tok@mi\endcsname{\def\PYG@tc##1{\textcolor[rgb]{0.13,0.50,0.31}{##1}}}
\expandafter\def\csname PYG@tok@kn\endcsname{\let\PYG@bf=\textbf\def\PYG@tc##1{\textcolor[rgb]{0.00,0.44,0.13}{##1}}}
\expandafter\def\csname PYG@tok@o\endcsname{\def\PYG@tc##1{\textcolor[rgb]{0.40,0.40,0.40}{##1}}}
\expandafter\def\csname PYG@tok@kr\endcsname{\let\PYG@bf=\textbf\def\PYG@tc##1{\textcolor[rgb]{0.00,0.44,0.13}{##1}}}
\expandafter\def\csname PYG@tok@s\endcsname{\def\PYG@tc##1{\textcolor[rgb]{0.25,0.44,0.63}{##1}}}
\expandafter\def\csname PYG@tok@kp\endcsname{\def\PYG@tc##1{\textcolor[rgb]{0.00,0.44,0.13}{##1}}}
\expandafter\def\csname PYG@tok@w\endcsname{\def\PYG@tc##1{\textcolor[rgb]{0.73,0.73,0.73}{##1}}}
\expandafter\def\csname PYG@tok@kt\endcsname{\def\PYG@tc##1{\textcolor[rgb]{0.56,0.13,0.00}{##1}}}
\expandafter\def\csname PYG@tok@sc\endcsname{\def\PYG@tc##1{\textcolor[rgb]{0.25,0.44,0.63}{##1}}}
\expandafter\def\csname PYG@tok@sb\endcsname{\def\PYG@tc##1{\textcolor[rgb]{0.25,0.44,0.63}{##1}}}
\expandafter\def\csname PYG@tok@k\endcsname{\let\PYG@bf=\textbf\def\PYG@tc##1{\textcolor[rgb]{0.00,0.44,0.13}{##1}}}
\expandafter\def\csname PYG@tok@se\endcsname{\let\PYG@bf=\textbf\def\PYG@tc##1{\textcolor[rgb]{0.25,0.44,0.63}{##1}}}
\expandafter\def\csname PYG@tok@sd\endcsname{\let\PYG@it=\textit\def\PYG@tc##1{\textcolor[rgb]{0.25,0.44,0.63}{##1}}}

\def\PYGZbs{\char`\\}
\def\PYGZus{\char`\_}
\def\PYGZob{\char`\{}
\def\PYGZcb{\char`\}}
\def\PYGZca{\char`\^}
\def\PYGZam{\char`\&}
\def\PYGZlt{\char`\<}
\def\PYGZgt{\char`\>}
\def\PYGZsh{\char`\#}
\def\PYGZpc{\char`\%}
\def\PYGZdl{\char`\$}
\def\PYGZhy{\char`\-}
\def\PYGZsq{\char`\'}
\def\PYGZdq{\char`\"}
\def\PYGZti{\char`\~}
% for compatibility with earlier versions
\def\PYGZat{@}
\def\PYGZlb{[}
\def\PYGZrb{]}
\makeatother

\begin{document}

\maketitle
\tableofcontents
\phantomsection\label{index::doc}


本文档主要参考自《工程设计学 学习与实践手册》:

\begin{Verbatim}[commandchars=\\\{\}]
git clone https://github.com/sinacloud/sae-python-dev-guide.git
\end{Verbatim}

Contents:


\chapter{第1章 引言}
\label{unit1::doc}\label{unit1:id1}

\section{设计领域}
\label{unit1:id2}

\subsection{任务和活动}
\label{unit1:id3}
设计是一种工程活动,它
\begin{itemize}
\item {} 
几乎触及到人类生活的全部领域,

\item {} 
应用自然科学的知识和法则,并且

\item {} 
以获得物质的实现为前提。

\end{itemize}

从工作心理学的角度看,设计是一种创造性的智力活动。

从方法学的观点看,设计是按给定的目标和在部分地相互矛盾的条件下求优过程。

从组织学的角度看,设计是创造价值过程中的主要部分,
是对毛柸进行精制的过程和对产品品质精心提高的过程的主要部分。

根据要求和有关的进程方式及工作方式,设计者可以在组织上以各种不同的方式参加到工作流程和生产流程中去:
\begin{itemize}
\item {} 
首先,各个公司愈来愈多地考虑各自负责的经营范围,按产品组来划界。

\item {} 
在没有具体客户委托的新设计中,方案设计和技术设计工作,常常与任务完成后的交付工作从组织上分类开来。

\item {} 
在大量的或多或少是一次性的项目中,委托任务往往是适应性设计方式下的继续开发任务。

\item {} 
在小型机器和仪器制造中由于有较多的零件数,往往宜于把实验室也划归设计部门。

\end{itemize}

设计领域的组织形式,常受下述情况的影响:
\begin{itemize}
\item {} 
各个具体化阶段的分步开发:通过反复的开发和设计过程,进行批量产品开发的优化过程。

\item {} 
一个合成产品的部件的平行开发:对于较大型的和大规模的产品,

\end{itemize}

不可避免地要按可分解的部件或功能单元来进行分部开发。
* 数据处理的采用:其中特别是CAD的可能性,将使未来的工作结构发生巨大的变化。


\subsection{设计的类型}
\label{unit1:id4}
\textbf{新设计}
\begin{quote}

按照一个系统(装置、设备、机器或部件)的相同的、有变化的或是新的任务书的要求,制定一个新的解的原理。
\end{quote}

\textbf{适应性设计}
\begin{quote}

使一个已有的系统(解的原理保持不变),适应于一个有改变的任务书,以克服原有系统的明显的局限性,这时常常必须对个别部件或零件作新设计。
\end{quote}

\textbf{变型设计}
\begin{quote}

在预先选定的系统范围内作尺寸和(或)布置方式的变型,功能和解的原理保持不变。
\end{quote}


\subsection{方法学设计的要求和必要性}
\label{unit1:id5}
要提出一种设计方法学,那么它应该具有下列特点:
\begin{itemize}
\item {} 
使一个针对问题的进程成为可能。

\item {} 
促进发明和认识能力。

\item {} 
同其它学科的概念、方法和知识兼容。

\item {} 
不靠偶然性求解。

\item {} 
所得的解易于移植到其它同类的任务中去。

\item {} 
适用于引入电子数字处理技术装置。

\item {} 
可教、科学。

\item {} 
符合工作科学的原理。

\end{itemize}


\section{方法学设计的发展}
\label{unit1:id6}

\subsection{历史回顾}
\label{unit1:id7}

\subsection{设计方法}
\label{unit1:id8}

\section{系统技术的方法}
\label{unit1:id9}
系统技术方法的一个重要的应用领域是基于功能的综合。在一个已知的或者已开发的解的方案的基础上,
开发一个功能模型或结构模型(功能结构、组合链),它的输入量和输出量及其联接,
通过计算方法的变异而按愿望发生变化,并按任务说明书的要求被优化。


\chapter{第2章 基础}
\label{unit2::doc}\label{unit2:id1}

\section{机器构造系统的基础}
\label{unit2:id2}

\subsection{系统、装置、设备、机器、仪器、部件、零件}
\label{unit2:id3}
技术任务是依靠技术产物来完成的。技术产物也就是通称的装置、设备、机器、仪器、部件、机器零件
或单件(按复杂程度来排列)。

一部机器是由部件和零件组成。将转换能量的技术产物称为机器,转换物料的技术产物称为设备,
转换信号的技术产物称为仪器。

胡勃卡将技术产物看成系统,系统与外界环境通过输入量(Inputs)和输出量(Outputs)而发生联系。
系统可以分解成若干分系统,通过系统边界来确定哪些东西属于哪个系统。输入量和输出量则跨越系统边界。


\subsection{能量、物料和信号的转换}
\label{unit2:id4}
韦茨舍克将能量、物质和信息这三个概念并列为基本概念。
如果涉及到它们的变化,亦即如果是处在流动之中,就必须与时间这一基本量发生关系。

\begin{Verbatim}[commandchars=\\\{\}]
能量:机械能、热能、电能、化学能、光能... ...
物料:气体、液体、固体... ...
信号:测量值、显示值、控制脉冲... ...
\end{Verbatim}

对于上述各种量的每一个转换,都必须关注其数量和质量。

在技术系统中发生着能量、物料和(或)信号的转换,必须用数量、质量和成本方面的说明来加以精确规定。


\subsection{功能关系}
\label{unit2:id5}
在一个系统中建立输入和输出间的唯一确定而能再现的相互关系。

功能:来代表一个系统的输入和输出之间,以完成任务为目的的总的相互关系。

如果总任务的规定已足够精确,即所有参与的量和它们在输入和输出方面所具有或需要的性质都已经知道,
总功能也就可以说明。

在许多情况下,总功能可以分解为若干可辨别的分功能,对应于总任务内的若干分任务。

将分功能有意义地、相容地联接成总功能,就引出了功能结构。

主功能是直接服务于总功能的那些分功能。副功能知识作为佐援性的功能而间接地对总功能做出贡献。


\subsection{作用关系}
\label{unit2:id6}
建立功能结构会使求解容易进行,因为结构化降低了工作的复杂程度,而且可以先单独地制定各个分功能的解。

物理事件依靠物理效应的存在,并通过确定几何的和物料的特征标志而发生于作用关系之中,
这一作用关系促使由任务书所限定的功能得以实现。

作用关系系由选用的物理效应和确定的几何和物料特征标志所决定的:
\begin{enumerate}
\item {} 
物理效应

\end{enumerate}
\begin{quote}

物理效应可以通过将有关的量相互联系起来的物理定律来描述,包括定量描述。
\end{quote}
\begin{enumerate}
\setcounter{enumi}{1}
\item {} 
几何的和物料的特征标志

\end{enumerate}
\begin{quote}

物理事件发生作用的地方称为作用地点。就在此处,通过作用几何,
亦即通过安排作用面(或作用线、作用空间)和选择作用运动,来相应的物理效应促使功能实现。
\end{quote}


\subsection{组合关系}
\label{unit2:id7}
组合结构考虑了制造、安装、运输等等的需要。从这种关系中就产生了构件、部件或机器及其有关的连接,
它们最终地构成了具体的技术产物和系统。


\subsection{系统关系}
\label{unit2:id8}
有效作用:功能性的作用,它是希望有的致用效果。

施入作用:功能性的联系,它是人在技术系统中的行为。

反作用:技术产物对人或对另一技术产物的功能性联系。

干扰作用:由外界作用于技术系统、技术产物或人的、在功能上不合乎需要的影响,
它有损于或有碍于实现功能。

副作用:由技术产物或系统作用于人和环境的、在功能上不希望有和不合乎意图的作用。


\subsection{一般的目标和条件}
\label{unit2:id9}
技术任务的解是通过要达到的目标和通过限制条件来确定的。

计数功能的满足,它的合乎经济的实现,以及保持人和环境的安全,这些可看作是一般的设置目标。


\section{方法学进程的基础}
\label{unit2:id10}

\subsection{一般的工作方法学}
\label{unit2:id11}
采用方法学进程时,必须满足下列前提:

\begin{Verbatim}[commandchars=\\\{\}]
确定目标:通过明确总目标、各个分目标及其重要性来达到。由此确保推动任务的解决,并增强自己的洞察力。

指出条件:亦即明确边界条件和初始条件。

消除偏见:只有这样才能在广泛范围内寻求解并避免思维差错。

寻求变型:亦即总是要找到更多的解,然后可由此选出最佳者。

评价:以目标和给定条件为考虑依据。

做出判断:借助于前述的评价,不难做出决断。没有决断,便不可能有认识上的前进。
\end{Verbatim}
\begin{enumerate}
\item {} 
直觉思维和逻辑推理思维

直觉的思维和进程,在进行时带有很强烈的忽发性,并且很突然地感觉到大觉大悟,
而对这种觉悟是无法施加影响或再现的。

纯粹直觉式的工作方式有缺点,所以要力求实行一种有意识的进程,它能按步就班地处理待解的问题。
这种工作方式称为逻辑推理的工作方式。

\item {} 
分析过程

分析,是通过分解、离析以及考察各个要素的性质和相互关系来获得信息的一种特殊方法。

问题分析就是要将本质和非本质的东西区分开来,当提出的问题比较复杂时,要分解成若干一目了然的分问题,
以准备用逻辑推理方式求解。

结构分析就是寻求结构关系。

弱点分析,分析原理设计方案和技术设计方案的弱点,并设法改进。

\item {} 
抽象过程

依靠抽象化,在定义问题时,可避免其形成时或使用时的偶然性,从而可导致普遍适用的解。

\item {} 
综合过程

综合就是通过建立连结,通过将若干元素相互结合以获得总体上新的作用,并通过揭示总括性的次序排列,
来加工信息。

综合时一般建议采用所谓的整体思维或系统思维。

\item {} 
通用的方法

一些常用方法:

\begin{Verbatim}[commandchars=\\\{\}]
有目的提问法

否定和更新方案法

前进法

后退法

因子化法

系统化法

分工和合作
\end{Verbatim}

\end{enumerate}


\subsection{作为信息转换的解决过程}
\label{unit2:id12}\begin{enumerate}
\item {} 
信息转换

在一个解决过程中始终存在对信息的需求,其中要获取、加工和输出信息,称为信息转换。

信息的获取

信息的加工

信息的输出

信息的贮存

\item {} 
信息系统

\end{enumerate}


\chapter{第3章 设计过程}
\label{unit3::doc}\label{unit3:id1}

\section{一般的解决过程}
\label{unit3:id2}
把工作步骤和决断步骤分开,保证了在设定目标、做计划、执行(组织)和检验之间存在必要而不可解脱的相互关系。

每一个任务书首先造成了一次对峙,即问题与已知的或未知的实现可能性之间的对立。

接着,在抽象阶面上定义本质问题,以此可确定目标并描述本质性条件。

接下来就是真正的创造性阶段——创造,此时要按照各种解题方法来开发解题主意,并借助于方法学规则将其变异和组合。

评价,这是赖以决断出显得较佳的变型的依据。

通过决断,得到原则性的意见。

在设计过程的不同部位,要累次重复从对峙经过创造直到决断这样的完整流程,并且总是在求解的不同具体化等级上进行。


\section{设计时的工作流程}
\label{unit3:id3}
七个基本的工作阶段:从信息开始经过拟定任务书,考察功能,制定原理解,建立模块式组合结构,直到总体结构设计,
最后完成产品技术文件

主要阶段为:

\begin{Verbatim}[commandchars=\\\{\}]
阐明任务书           即信息方面的确定
方案设计             原理方面的确定
技术设计             结构方面的确定
施工设计             制造技术方面的确定
\end{Verbatim}
\begin{description}
\item[{\textbf{阐明任务书}:}] \leavevmode
用它来获得下列方面的信息:对解提出的要求、存在的条件及其重要性。

本步工作的结果是以一张要求表作信息方面的确定。

\item[{\textbf{方案设计}:}] \leavevmode
在阐明任务书后,通过抽象化来认识本质问题,建立功能结构,
并通过寻求合适的作用原理且将其组合成作用结构,来确定原理解。方案设计是在原理方面确定一个解。

\item[{\textbf{技术设计}:}] \leavevmode
对于一个技术产品,从作用结构或原理解出发,根据技术观点和经济观点来明确地、完整地拟定组合结构。
技术设计是在结构方面确定一个解。

\item[{\textbf{施工设计}:}] \leavevmode
通过最终地规定所有零件的形状、尺寸和表面质量,确定所有材料,检验制造可能性及最终成本,
来补充一个技术产物的组合结构,并完成其在物质上实现所需的有关图纸和其它资料。

施工设计的结果是在制造技术方面确定解。

\end{description}


\chapter{第4章 制定产品计划和阐明任务}
\label{unit4::doc}\label{unit4:id1}

\section{制定产品计划}
\label{unit4:id2}

\subsection{任务和进程}
\label{unit4:id3}
根据企业目标制定产品计划,包含了系统地寻求和选择有前途的产品主意并对其跟踪。

指定产品计划的原发性推动力可以是外界,即通过市场和周围环境而产生,也可以通过企业本身而产生。

来自市场的推动力主要有:
\begin{itemize}
\item {} 
本企业产品的技术和经济性态。

\item {} 
市场需求的变化。

\item {} 
顾客的鼓励和批评。

\item {} 
竞争者产品的技术和经济优点。

\end{itemize}

来自周围环境的推动力主要有:
\begin{itemize}
\item {} 
经济政治事变的发生。

\item {} 
被新工艺和研究结果所替代。

\item {} 
现有产品和工艺过程的防污染化及再循环。

\end{itemize}

来自企业自身的推动力主要有:
\begin{itemize}
\item {} 
在开发和制造中采用自己的注意和研究成果。

\item {} 
用以扩大或满足销售范围的新的功能。

\item {} 
引入新的制造方法。

\item {} 
产品配套和制造结构上的合理化措施。

\item {} 
参加可能性的利用。

\item {} 
更高的多品种程度。

\end{itemize}


\subsection{形势分析}
\label{unit4:id4}
了解自身的形势,要进行若干项考察:
\begin{itemize}
\item {} 
根据销售、赢利和偿还款项的有关数据。

\item {} 
依靠所述的产品-市场矩阵,了解产品在目前市场上所处地位。

\item {} 
了解自身在技术上的不完整性。

\item {} 
明确技术现状。

\item {} 
通过标准、国际推荐、指导性文件、规程来掌握硬性的数据和重点。

\item {} 
通过新的规划项目、趋势和用户意见来估计将来的发展。

\end{itemize}

形式分析确定了搜索策略和要处理的搜索区。


\subsection{建立搜索策略}
\label{unit4:id5}
综合考虑企业目标、企业优势和周围环境后,可能找到一个有利于借以确定搜索区的缺档。

寻找合适的搜索区时,可借助于需求-优势矩阵。


\subsection{找出产品主意}
\label{unit4:id6}

\subsection{选择产品主意}
\label{unit4:id7}
所获得的产品主意,先经过一个挑选过程。


\subsection{定义产品}
\label{unit4:id8}

\subsection{产品建议}
\label{unit4:id9}
产品建议应当:
\begin{itemize}
\item {} 
描述那些意欲实现的功能。

\item {} 
包含一张初步的要求表。

\item {} 
不偏向于某种解地撰文叙述读新产品的所有要求。

\item {} 
载明与企业目标相关联的成本目标或成本范围,并清楚地说明将来的意向。

\end{itemize}


\section{阐明任务书}
\label{unit4:id10}

\subsection{一份阐明了的任务书的重要性}
\label{unit4:id11}
任务书一般以下列形式提交设计或开发部门:
\begin{itemize}
\item {} 
作为开发任务,

\item {} 
作为具体订货

\item {} 
作为通过销售、试验、测试、装配或相邻设计部门及本设计部门而来的改进建议和批评之类所得到的推动激励。

\end{itemize}

必须阐明:
\begin{itemize}
\item {} 
真正要处理的问题是什么?

\item {} 
存在哪些往往未表达出来的愿望和期望?

\item {} 
在任务书中给出的条件是否真实?

\item {} 
开发工作中有哪些可能途径?

\end{itemize}

下列问题是有用的:
\begin{itemize}
\item {} 
意向中的解必须满足什么目的?

\item {} 
它必须具备哪些特征?

\item {} 
它不允许有哪些特征?

\end{itemize}


\subsection{要求表}
\label{unit4:id12}\begin{enumerate}
\item {} 
内容

\end{enumerate}
\begin{quote}

为了编写要求表,必须用必达要求和愿望的方式来提出各种要求:
\begin{itemize}
\item {} 
必达要求,在各种情况下都必须满足的。

\item {} 
愿望,这是应尽可能考虑的,考虑时可能要做些妥协。

\end{itemize}

在不先确定某个解答的前提下,要在数量和质量方面把必达要求和愿望表达清楚。

\textbf{数量}:关于数目、件数、批量和数量的全部说明,单位的时间值。

\textbf{质量}:关于容许偏差和特殊要求的全部说明。
\end{quote}
\begin{enumerate}
\setcounter{enumi}{1}
\item {} 
结构

\end{enumerate}
\begin{quote}

要求表的结构。
\end{quote}
\begin{enumerate}
\setcounter{enumi}{2}
\item {} 
要求表的建立

\end{enumerate}
\begin{quote}

要求表的建立应按下列指示进行:

\begin{Verbatim}[commandchars=\\\{\}]
1.收集要求
2.仔细的排列各项要求
3.在格式纸上建立要求表
4.对于改动和补充进行校核,在要求表上予以补入
\end{Verbatim}
\end{quote}
\begin{enumerate}
\setcounter{enumi}{3}
\item {} 
例子

\item {} 
其它应用

\end{enumerate}


\chapter{第5章 方案设计}
\label{unit5::doc}\label{unit5:id1}
方案设计:在阐明了任务书以后,通过抽象化、建立功能结构、寻求合适的作用原理并将其组合,而确定原理解。


\section{方案设计的工作步骤}
\label{unit5:id2}
在阐明任务书后即进入方案设计阶段。


\section{抽象化以认清本质}
\label{unit5:id3}

\subsection{抽象的目的}
\label{unit5:id4}
在抽象化时,略去了个性和偶然性,而企求突出普遍适用的和本质的东西。


\subsection{抽象化和课题表述}
\label{unit5:id5}
一个任务所具有的普遍意义和本质性的东西,可通过功能关系和与任务相关的主要约束条件方面的分析,
并从要求表中一步一步地抽象化来获得:

\begin{Verbatim}[commandchars=\\\{\}]
第1步:在思想上撇开愿望;
第2步:撇开不直接涉及功能和本质性约束条件的必达要求;
第3步:将定量的说明改成定性的,从而简缩为本质性陈述;
第4步:有意义地扩大认识;
第5步:不偏向于某种解地将课题表述成文。
\end{Verbatim}

本步骤的结果是在抽象阶面上定义目标,而不确定具体的某种解。


\subsection{有系统地扩展课题表述}
\label{unit5:id6}
这一进程的特点是:将课题的表述一步一步地扩展到尽可能宽的程度。

抽象化的进程有助于识别虚假的限制条件,只让真正的条件保留下来。


\section{建立功能结构}
\label{unit5:id7}

\subsection{总功能}
\label{unit5:id8}
如果表述总任务时抓住了核心,也就可表达出总功能,
它就用框图表示能量、物料、和(或)信号转换方面输入量和输出量间的关系,而不偏向某种解。


\subsection{分解为分功能}
\label{unit5:id9}
任务之复杂程度指的是这种关系中输入和输出间的关系可以看清楚到什么程度,
所需物理过程有多少层次,以及预期中的部件和零件的数目有多大。

任务所需要的总功能可以分解成分功能,将各个功能结合起来,就得到功能结构,就表达了总功能。

当前这一工作步骤的目标是:
\begin{itemize}
\item {} 
将所需要的总功能分解为分功能,以使接着求解较为容易;

\item {} 
将这些分功能结合成简单、明确的功能结构。

\end{itemize}

在地道的新设计中,通常既不知道单个分功能,也不知道它们是如何结合的。

在适应性设计时,结构组成及其部件和零件在很大程度上是已知的。可以通过进一步开发的产品进行分析,来建立功能结构。


\subsection{逻辑关系考察}
\label{unit5:id10}
与-功能、或-功能和非-功能以及它们组合成的复合功能如非或-功能(带有非的或)、非与-功能(带有非的与)、
或借助双稳态的存贮功能,这些称为逻辑功能。


\subsection{物理关系考察}
\label{unit5:id11}
物理关系是通过将各个有关物理量结合起来的一个相应功能结构来表达的。

应当把结构中凡是明确存在的主流首先建立起来,以便进一步求解时考虑副流。


\subsection{功能结构实践}
\label{unit5:id12}
建立功能结构时,要区分新设计和适应性设计。

新设计时,功能结构的出发点是要求表和抽象的课题表述。

在适应性设计这种形式的继续开发工作中,是通过分析结构元件,从已知解得到功能结构作为初始开端。


\section{寻求作用原理}
\label{unit5:id13}
作用原理包含为实现一个功能所需的物理效应及几何和物料特征标志。


\subsection{传统的辅助手段}
\label{unit5:id14}\begin{enumerate}
\item {} 
\textbf{查阅文献}

\end{enumerate}
\begin{quote}

技术状况方面的信息,可以从来自多方面的专业书刊、从查阅专利、以及从竞争者产品的说明而获得。
\end{quote}
\begin{enumerate}
\setcounter{enumi}{1}
\item {} 
\textbf{分析自然系统}

\end{enumerate}
\begin{quote}

研究自然界中的形状、结构、生物和过程,以及利用生物学中获得的知识,可以引出有多方面用途而技术上新颖的解。
\end{quote}
\begin{enumerate}
\setcounter{enumi}{2}
\item {} 
\textbf{分析已知技术系统}

\end{enumerate}
\begin{quote}

可借以一步一步地并且可以重现地从已知解导出新的或改进的变型。
\end{quote}
\begin{enumerate}
\setcounter{enumi}{3}
\item {} 
\textbf{类比考察}

\end{enumerate}
\begin{quote}

将现有的课题或意想中的系统转移到类比物上去。
\end{quote}
\begin{enumerate}
\setcounter{enumi}{4}
\item {} 
\textbf{测量、模型试验}

\end{enumerate}
\begin{quote}

在已做成的系统上进行测量,利用相似力学进行模型试验,以及其它的实验研究。
\end{quote}


\subsection{偏重于直觉的方法}
\label{unit5:id15}
纯粹的直觉工作方式有下列缺点:
\begin{itemize}
\item {} 
正确的突发思想不在恰当时刻到来,因为它是不能被强迫来的。

\item {} 
由于现有的传统和自己僵化的先入之见,认识不到新的途径。

\item {} 
由于信息不够,新工艺或新流程没有渗透到设计师的感悟之中。

\begin{Verbatim}[commandchars=\\\{\}]
1.智暴法
2.635法
3.陈列法
4.德尔菲法
5.联想法
6.复合应用
\end{Verbatim}

\end{itemize}


\subsection{偏重逻辑思维的方法}
\label{unit5:id16}
通过有意识按步骤的进程来获得解。
\begin{enumerate}
\item {} 
系统地研究物理事件

\item {} 
依靠编排表式进行系统的寻求

\end{enumerate}
\begin{quote}

一方面,编排表式会启发人们沿某些方向寻求进一步的解;
另一方面,易于认识本质性的解的特征标志和相应的联接可能性。

编排依据或其参数的选择,具有决定性意义。
\end{quote}
\begin{enumerate}
\setcounter{enumi}{2}
\item {} 
目录的应用

\end{enumerate}
\begin{quote}

目录是一定设计任务或分功能的已知解或经过考验的解的汇编。

\textbf{对设计目录的要求}:
\begin{itemize}
\item {} 
可快速地根据任务的需要来检索目录中汇编的解或数据;

\item {} 
汇编入的解谱要充分完备,至少是可以补全的;

\item {} 
尽可能完全独立于部门或工厂,以使其广泛可用;

\item {} 
既可用于传统的设计过程,亦可用于采用计算机时。

\end{itemize}
\end{quote}


\section{作用原理的组合}
\label{unit5:id17}
为了实现任务书中所要求的总功能,必须从解(作用原理)场通过组合成为作用结构的方法来制定总解(系统综合)。

这个组合步骤的主要问题,是识别各个需要连接起来的作用原理之间在物理方面是否相容,
以期达到不受干扰之累的能量流、物料流和(或)信号流,以及识别几何关系上是否不发生冲突。
还有一个问题,是从理论上可能的组合场中选出技术上和经济上良好的组合。


\subsection{有系统的组合}
\label{unit5:id18}
用这种表式来拟定总解,需要为每个分功能选出一个作用原理,并且按功能结构中的次序把它们依次结合成总解。

这一方法的主要问题,是要决断出哪些作用原理是彼此相容并且不会冲突的,亦即真正可以组合起来的。


\subsection{依靠数学方法的组合}
\label{unit5:id19}
依靠数学方法由分解组合成总解时,必须知道分解的一些特征标志和性质,它们应与要连结的相邻解的相应性质相一致。


\section{挑选合适的变型}
\label{unit5:id20}
依靠一个有规则又可检验的选择过程,可便于从众多解的建议中做出选择。这样一种选择过程由淘汰和优选这两项工作来标志:

首先淘汰那些绝对不合适者。剩留下来的可能解如果仍太多,则优选那些显然较好者。


\section{具体化为原理解的变型}
\label{unit5:id21}

\section{方案变型的评价}
\label{unit5:id22}
对已具体化为原理解变型的解决建议进行评判,以获得客观的决断基础。


\subsection{基础}
\label{unit5:id23}
评价是要查明一个解关于预定目标的”价值“及”效用“或”优度“。
\begin{enumerate}
\item {} 
\textbf{弄清评价准则}

\end{enumerate}
\begin{quote}

评价的第一步是建立设定目标,由此可导出评价准则,并可按之评判解的变型。

对于技术任务,这些目标首先可从要求表的要求,以及从常常要与制定的解联系起来方可知道的一般条件而得到。

建立目标时,必须尽可能彻底满足下列前提条件:
\begin{itemize}
\item {} 
目标中应当尽可能完备地包括对决断有重大关系的要求和一般条件,以免在评价时忽略重要依据。

\item {} 
进行评价时所依据的各个目标,必须彻底相互独立。

\item {} 
在获取信息时所需费用不超过容许范围的条件下,需评价的系统关于各个目标的特性应尽可能定量地具体表达,至少也要定性地(用文字)具体表达。

\end{itemize}
\end{quote}
\begin{enumerate}
\setcounter{enumi}{1}
\item {} 
\textbf{研究对于总价值的重要性}

\end{enumerate}
\begin{quote}

建立评价准则时,必须弄清楚他们对于一个解的总价值的重要性(权值),
从而必要时可在真正评价以前就去掉不重要的评价准则。
\end{quote}
\begin{enumerate}
\setcounter{enumi}{2}
\item {} 
\textbf{特性值的汇编}

\end{enumerate}
\begin{quote}

建立了评价准则并确定了它们的重要性后,在下一个工作步骤中,
需要对需评价的解的变型补列上已知的或通过分析得出的特性值。
\end{quote}
\begin{enumerate}
\setcounter{enumi}{3}
\item {} 
\textbf{按价值观念评价}

\end{enumerate}
\begin{quote}

通过引入评价者的价值观念,来从已经定出的特征值得出”价值“。

对于价值确定法必须明确:不论是建立价值函数还是建立评分表式,都可能强烈地受到主观影响。
\end{quote}
\begin{enumerate}
\setcounter{enumi}{4}
\item {} 
\textbf{确定总价值}

\end{enumerate}
\begin{quote}

将各个分价值总加起来以评价技术产品。
\end{quote}
\begin{enumerate}
\setcounter{enumi}{5}
\item {} 
\textbf{比较解的变型}

\end{enumerate}
\begin{quote}

以加法规则为基础,对变型的评价可能做法:

确定最大总价值:用这个方法时,将总价值最大的那个变型判为最佳。

确定一个价值比:将总价值与一个想象的,由最大可能的价值得出的理想价值相比。

粗略比较解的变型
\end{quote}
\begin{enumerate}
\setcounter{enumi}{6}
\item {} 
\textbf{对评价不可靠性的估计}

\end{enumerate}
\begin{quote}

上面建议的评价方法所可能有的误差或不可靠性可分成两个主要类别,
即:由人引起的评分误差和方法本身带来的原理性缺陷。
\end{quote}
\begin{enumerate}
\setcounter{enumi}{7}
\item {} 
\textbf{寻找弱点}

\end{enumerate}
\begin{quote}

从关于某个评价准则的价值低于平均水平,即可看出弱点来。
\end{quote}


\subsection{评价法的比较}
\label{unit5:id24}

\subsection{方案设计阶段的评价实践}
\label{unit5:id25}
在方案设计阶段中,单个分步骤应如下安排:
\begin{quote}

弄清楚评价准则

评价准则系从下列因素获得:
\begin{enumerate}
\item {} 
要求表的要求

\item {} 
一般的技术和经济特性

\end{enumerate}

对于总价值的重要性(加权)

特性值的汇编

按价值观念评价

确定总价值

比较解的变型

估计评判的不可靠性

寻找弱点
\end{quote}


\section{方案设计例子}
\label{unit5:id26}

\chapter{第6章 技术设计}
\label{unit6::doc}\label{unit6:id1}
技术设计的内容是由作用结构或原理解出发,
按照技术和经济的观点,明确、完整地确定技术产品的组合结构。技术设计的结果就是解的方案的结构确定。


\section{技术设计的工作步骤}
\label{unit6:id2}
技术设计过程的复杂性:
\begin{itemize}
\item {} 
许多工作必须同时平行进行,

\item {} 
一些工作步骤要在更高的信息级别上重复,

\item {} 
已经构形的区域受到补充和变化因素的影响,

\end{itemize}

技术设计进程:
\begin{enumerate}
\item {} 
首先一步是从要求表、必要时由其摘要出发,在了解原理解(作用结构、方案)的基础上,制定出主要的决定结构的要求:

\end{enumerate}
\begin{itemize}
\item {} 
决定尺寸的要求,

\item {} 
决定布置的要求,

\item {} 
决定材料的要求,

\end{itemize}
\begin{enumerate}
\setcounter{enumi}{1}
\item {} 
说明决定和限制结构设计的空间条件。

\item {} 
在明确了带有空间条件的决定结构的要求以后,初选材料,进行组合结构的粗鲁构形。

\item {} 
首先对确定结构的主功能载体进行粗略构形,也就是初步地估计其材料与形状。

\item {} 
按照之前介绍的方法,对此时的技术设计草案进行判断。

\item {} 
对由于是已知的、确定了的、次要的、或至今未确定结构等原因而尚未研究过的主功能载体,在要求的范围内补作粗略结构设计。

\item {} 
确定哪些辅助功能是必要的,并且利用现有的解。

\item {} 
对主功能载体进行精确结构设计。

\item {} 
对辅助功能载体也进行精确的结构设计。

\item {} 
按照技术与经济准则进行评价。

\item {} 
确定初步的总体技术设计方案。

\item {} 
消除评价时已发现的弱点。

\item {} 
检查总体技术设计方案在功能、空间相容性等方面的缺陷及干扰因素,并在必要时加以改善。

\item {} 
提出初步的零件表以及加工和装配说明书,从而完成最后有效的总体技术设计方案。

\item {} 
确定总体技术设计方案,并且提交给施工设计。

\end{enumerate}


\section{结构设计的导则}
\label{unit6:id3}
结构设计的特征就是一个不断反复的思考和检查的过程。

在每个结构设计的过程中,首先是通过与选择材料同时进行的参数计算去实现功能。

经常采用初步计算的方法,得到初始的、按比例的图形,并可对空间相容性做粗略判断。


\section{结构设计的基本规则}
\label{unit6:id4}
基本规则明确、简单和安全是从下面一般的目标设置推导出来的,即:
\begin{itemize}
\item {} 
满足技术功能,

\item {} 
经济地实现设计目标,

\item {} 
对人和环境均安全。

\end{itemize}

对明确性的注意,有助于可靠地预先确定作用和性能,并且在很多情况下节省时间和昂贵的研究费用。

简单性保证在正常情况下获得一个经济的解的方案。较少的零件数目和简单的结构型式,
可以使加工较快、较好地进行。

对安全性的要求,促使人们坚持根据耐用性、可靠性、消除事故以及保护环境等方面去处理问题。


\subsection{明确}
\label{unit6:id5}
基本规则“明确”应用于:

\textbf{功能}:在一个功能结构内部必须
\begin{itemize}
\item {} 
建立带有各自特有的输入和输出量的分功能的明确排列关系。

\end{itemize}

\textbf{作用原理}:所选出的作用原理在物理效应方面应该能够
\begin{itemize}
\item {} 
描述原因和效应之间的关系,从而能正确且经济地进行参数计算。其作用结构必须

\item {} 
确定能量流(或力流)、物料流和信号流的有次序的走向。

\end{itemize}

考虑到由载荷引起的变形和温度引起的膨胀,应当
\begin{itemize}
\item {} 
在结构上预先安排沿一定方向伸长的可能性。

\end{itemize}

\textbf{参数选择计算}:为了进行参数计算和材料选择,在大小、种类、
频繁度或者时间方面明确说明载荷状态是绝对必要的。

\textbf{安全性}

\textbf{人机工程}:在人-机关系方面应当使得
\begin{itemize}
\item {} 
操作顺序及其实施利用相应的布置和接入方式按照合理的步序进行。

\end{itemize}

\textbf{加工和检验}:在图纸、零件表和说明书中采用明确和完整的说明,可以使得加工和检验变得容易些。

\textbf{装配和运输}:通过结构的构形使得装配必然按顺序进行,而排除错误的工作次序。

\textbf{使用和维护}:应当注意明确的构造和相应的构形,使得:
\begin{itemize}
\item {} 
工作结果一目了然,并且易于检查,

\item {} 
维修能采用尽可能少的不同辅助材料及工具进行,

\item {} 
维修可进行检查。

\end{itemize}

\textbf{回用}:应规定
\begin{itemize}
\item {} 
明确的可回用材料的分离位置及

\item {} 
明确的装配及拆装顺序。

\end{itemize}


\subsection{简单}
\label{unit6:id6}
\textbf{功能}:
\begin{itemize}
\item {} 
尽可能少的分功能数目以及

\item {} 
其间的联接明了而合乎逻辑。

\end{itemize}

\textbf{作用原理}:
\begin{itemize}
\item {} 
作用过程和组成件的数目少,

\item {} 
规律清楚及

\item {} 
费用低廉。

\end{itemize}

\textbf{参数选择计算}:
\begin{itemize}
\item {} 
几何形状是基础,它对列出材料力学和弹性力学的数学计算式是直接有用的。

\item {} 
选择对称的形状,使得由加工、负载和温度引起的变形一目了然。

\end{itemize}

\textbf{安全性}:

\textbf{人机工程}:
\begin{itemize}
\item {} 
明确的操作过程,

\item {} 
一目了然的布置方式及

\item {} 
易于理解的信号

\end{itemize}

\textbf{加工和检验}:
\begin{itemize}
\item {} 
几何形状一般,能节省加工时间,

\item {} 
加工方法少,能减少夹紧、调整和等待的时间,

\item {} 
形状清楚,使得检验容易和迅速。

\end{itemize}

\textbf{装配和运输}:
\begin{itemize}
\item {} 
要装配的零件易于识别,

\item {} 
能够迅速地理解装配工艺,

\item {} 
每一个调整过程只要进行一次,

\item {} 
避免已装配零件的重复装配。

\end{itemize}

\textbf{使用和维护}:
\begin{itemize}
\item {} 
用不着特别的和复杂的说明就可以使用,

\item {} 
过程一目了然,对差错或干扰易于觉察,

\item {} 
如果维修过程麻烦、不舒适和费时间,则停止进行。

\end{itemize}

\textbf{回用}:
\begin{itemize}
\item {} 
采用可回用的材料,

\item {} 
装配和拆卸简单,

\item {} 
部件本身简单

\end{itemize}


\subsection{安全}
\label{unit6:id7}\begin{enumerate}
\item {} 
安全技术的概念、种类和范围

\end{enumerate}
\begin{itemize}
\item {} 
\textbf{安全性}:危险小于临界冒险度的一种事物状态。

\item {} 
\textbf{临界冒险度}:因设备而异的某一技术过程或技术状态可以承受的最大的危险程度。

\item {} 
\textbf{保护}:通过减少发生损伤的频繁度或损伤范围、或者同时减少这二者的相应预防措施,以减少冒险的程度。

\item {} 
\textbf{可靠性}:一个技术系统在给定的工作范围及规定的工作时间内满足由使用目的确定的工作要求的能力。

\item {} 
\textbf{可用性}:系统可正常提供使用的时间比例,可与日历时间或规定的额定时间相比。

\item {} 
\textbf{运行安全性}:包括技术系统运行时对危险的限制,使得系统本身及其直接环境不致损坏。

\item {} 
\textbf{工作安全性}:在工作时、即使用技术系统时,以及在工作环境之外、例如运动和休息时、对人的危害的限制。

\item {} 
\textbf{环境安全性}:对技术系统环境周围环境损害的限制。

\item {} 
\textbf{保护措施}:通过保护系统或保护措施限制已存在的危害,从而冒险性降低到可容许的程度。

\end{itemize}
\begin{enumerate}
\setcounter{enumi}{1}
\item {} 
直接安全技术原理

\end{enumerate}

直接安全技术要求借助参与本身工作的系统或构件获得安全性。

为了确定和判断功能的可靠实现及构件的耐用度,确定一种安全性原理,基本上有下述三种类型:

\begin{Verbatim}[commandchars=\\\{\}]
1.“保持安全”原理
2.“限制失效”原理
3.“冗余配置”原理
\end{Verbatim}

\textbf{保持安全原理}:对所有的构件及其相互关系应如此处理,
使得在规定的工作时间内能经受住所有的可能事件而不产生失效或干扰。

通过下述措施保证:
\begin{itemize}
\item {} 
相应地弄清楚作用的负荷和环境条件。

\item {} 
在可靠的假设和计算方法下的足够安全的计算。

\item {} 
加工和装配中足够的、彻底的检查。

\item {} 
在局部提高的载荷条件和当时的环境影响下,对构件或系统进行研究,以确定其耐用性。

\item {} 
规定使用范围,而将可能产生失效的范围排除在外。

\end{itemize}

\textbf{限制失效原理}:允许在使用期限内产生功能干扰和破裂,但是不允许发生严重的后果。在这种情况下,应当:
\begin{itemize}
\item {} 
仍保持有限的功能或能力,以避免出现危险的状态,

\item {} 
失效零件的有限的功能为其它零件所承担,直到设备或机器无危险地停止为止,

\item {} 
缺陷或失效能够觉察出来,

\item {} 
能够对失效部位作出对总的安全性起决定性作用的工作状态的判断。

\end{itemize}

限制失效原理是以对损坏过程的了解和失效时承担或者保持有限功能的某种结构方案作为先决条件的。

\textbf{冗余配置原理}:既能提高系统的安全性,又能提高其可靠性的一种手段。

积极的冗余:利用装置,在某单个大部件失效时,其功能也不会完全中断。
消极冗余:在积极单元发生失效时接入的、一般在种类和尺寸上相等的贮备单元。
原理冗余:多重配置在功能上一样,而作用原理不同。
\begin{enumerate}
\setcounter{enumi}{2}
\item {} 
间接安全技术原理

\end{enumerate}

间接安全技术包括保护系统和保护设施。

\textbf{保护系统}:在发生危险时引起保护反应。
此该系统应在一个具有信号转换的功能结构中至少有一个获知危险的输入量和一个能消除危险的输出量。

这种系统的作用结构是建立在具有获知、处理和发生作用等主要功能的功能结构的基础上的。

\textbf{保护机构}:以其自身的功能能力为基础,而无须信号转换就可以行使保护功能的技术结构。

\textbf{保护设施}:无须保护反应而具有保护功能。

\textbf{基本要求}:

实现保护技术的所有安全技术措施必须满足下列基本要求:
\begin{itemize}
\item {} 
作用可靠

\item {} 
强制有效

\item {} 
不能回避

\end{itemize}

\textbf{作用可靠意味着}:作用原理和结构形状只可能有一个确切的作用方式;
参与工作的部件按照可靠地规划进行计算;加工和装配在严格检查下进行;
保护系统和保护措施已经经过了样机试验。

\textbf{强制失效意味着}:
\begin{itemize}
\item {} 
在引起危险的情况开始时和进行过程中,都必须起作用,

\item {} 
当取消保护措施或保护设施时,引起危险的工作状态必须强迫停止。

\end{itemize}

\textbf{不能回避意味着}:既不能通过随机或非随机的变化,又不能通过外界干涉,
而使得保护作用遭受破坏或不起作用。

\textbf{保护系统}:

保护系统的任务:当存在危险时,自动产生保护反应,从而防止对人和物的损害。

当出现危险时,避免危险发展:
\begin{itemize}
\item {} 
机器或设备停止运行,

\item {} 
阻止启动。

\end{itemize}

当危险持续存在时,避免危险起作用:
\begin{itemize}
\item {} 
引入保护措施。

\end{itemize}

“可靠作用”、“强制生效”和“不能回避”的基本要求为下述要求所支持:

\textbf{发出信号}:在接入保护系统时,必须伴随一个信号,说明接入这一事实和机器断开的原因。

\textbf{自监控}:保护系统不仅应在危险状态下做出反应,而且在它本身存在妨碍正常保护作用的缺陷时,
也应做出反应。这一要求最好通过按照静流原理的计算达到。

\textbf{冗余}:保护系统的失效是一种可能发生的情况。双重或多重地设置保护系统可以提高其安全性,
这是由于不大可能所有设置的保护系统一次全部失效。

\textbf{双稳态性}:保护系统和保护机构必须安置在一定的起动阀值上。
一旦达到此值,立即毫不延缓地、明确地做出保护反应。这一性能是通过所谓双稳态性而强制发生的。

\textbf{防止重新起动}:在保护系统通过双稳态特性使机器断开后,不允许自行恢复到正常运行状态,
即使危险状态已不再存在也应如此。

\textbf{可检验性}:即使不存在危险状态的情况下,也应当能够检查保护系统的功能能力。

\textbf{降低要求}:只有在下述情况下,才可以有意识地降低要求,即产生失效的概率很低,
而且在产生危险的情况下破坏性很小,以至允许放弃某些要求。当保护系统的检验简单可行,
而且这样的检验可以按规律地强制进行,则在进一步考虑时,可以放弃对冗余方面的要求。

\textbf{保护设施}:
保护设施的任务是将人和物从危险的地方分离开来,并保护他们不受各种形式危险输出(作用)的损害。

所追求的原理解通过下列措施阻止接触:
\begin{itemize}
\item {} 
所有方向加罩,

\item {} 
在某一方向盖住以防接触,

\item {} 
在一定距离设置防护层。

\end{itemize}

由人体四肢及其可及范围决定的安全距离起着重要的作用。

4.安全技术的参数选择计算与检查

\textbf{功能和作用原理}

重要的问题是,采用选出的解能否安全和可靠地实现功能。必须同时考虑明显的和可能出现的干扰。

\textbf{参数选择和计算}

韧性,即塑性变形的能力,可以在应力分布不均匀时降低应力的峰值,
是材料能够提供给我们的一个重要安全因素。

稳定性,涉及所有的状态稳定和倾覆危险的问题,也涉及一个机器或装置的稳定运行问题。

共振会造成不可确切估计的应力提高。

注意热膨胀问题,以避免应力过大或者功能受到干扰。

\textbf{人机工程学和工作安全性}

起决定作用的是对危险根源和危险部位的了解。

\textbf{加工和检验}

零件的结构应能使得要求的质量特性也可能通过加工达到和遵守。
这一点通过相应的、在必要情况下以规章强制的检验加以保证的。

\textbf{装配和运输}

在技术设计阶段就应当知道和考虑到在装配时与强度和稳定性有关的负载。

\textbf{维修}

使用和操作必须尽量安全。

\textbf{成本和期限}

成本和期限的约束不允许对安全性产生影响。


\section{结构设计的原理}
\label{unit6:id8}

\subsection{力传导原理}
\label{unit6:id9}\begin{enumerate}
\item {} 
力流和等结构强度原理

\end{enumerate}
\begin{quote}

在机械制造以及精密机械中的任务和求解,差不多都是在物料、
能量和信号转换的相互关系上处理力和运动的产生以及其联结、转变、变化和导通的问题。

力传导的概念,包括弯矩和扭矩的传导。

以下情况是有好处的,即:

\begin{Verbatim}[commandchars=\\\{\}]
外载荷(作用在构件上的)影响
截面力(纵向力、横向力、弯矩和扭矩),在构件内部引起
应力(拉、压正应力,剪、扭切应力),并且造成
弹性或塑形变形(伸长、缩短、横向收缩、弯曲、剪切和扭转)。
\end{Verbatim}

等结构强度原理是通过合适地选择材料和形状力求在规定的时间内各处强度同样充分地得到应用。
\end{quote}
\begin{enumerate}
\setcounter{enumi}{1}
\item {} 
直接和短程力传导原理

\end{enumerate}
\begin{quote}

要求从某一位置到另一位置的力或力矩尽可能引起最小的变形,最合适的办法是采取直接的和最短的传力路线。

方法的选择主要取决于任务的性质:
\begin{itemize}
\item {} 
是否与力的传导有关,此时在构件刚度尽可能高的情况下其耐久性起决定性的作用,或者

\item {} 
是否必须满足要求的力-变形关系,而耐久性仅仅是一个附属的、值得注意的问题。

\end{itemize}

倘若已经超过屈服极限,应当考虑下述问题:
\begin{itemize}
\item {} 
如果结构受到力的作用,则产生相应的变形为必然的结果。

\item {} 
如果构件发生变形,则存在相应的反作用力。

\end{itemize}
\end{quote}
\begin{enumerate}
\setcounter{enumi}{2}
\item {} 
变形协调原理

\end{enumerate}
\begin{quote}

为了不出现具有尖峰应力的非均匀应力分布,
考虑力流观点的结构设计应力求避免由于急剧的截面过度而引起的突然的“力流转向”和“力流密度”的变化。

按照变形协调原理,应对参与工作的组件这样进行结构设计,
使得载荷作用时在尽可能小的相对变形情况下借助于相应的同向变形而得到更进一步的匹配。
\end{quote}
\begin{enumerate}
\setcounter{enumi}{3}
\item {} 
力的平衡原理

\end{enumerate}
\begin{quote}

按照主功能的定义,那些用于直接实现功能的力和力矩,可以看成是决定功能的力参量。

这些力或者力矩伴随主参量而生,并且固定地依附于它们,称它们为伴生的副参量。

总的来说,对于力的传导应当做到:
\begin{itemize}
\item {} 
力流持续封闭,避免由于

\item {} 
截面突然变化而引起的

\item {} 
力流转向和力流密度变化。

\end{itemize}

力流概念通过考虑下列各原理而加以补充:
\begin{itemize}
\item {} 
\textbf{等结构强度原理}:力争借助于选择合适的材料和形状使得在规定的工作期限内在所有各处强度同样地得到高度充分利用。

\item {} 
\textbf{直接和短程的力传导原理}:使得材料消耗、体积、重量和变形最小,特别适用于要求构件刚度大的情况。

\item {} 
\textbf{变形协调原理}:考虑由应力所引起的变形,寻找具有相互协调变形机理的布置方式,从而避免应力提高,并且可靠地实现其功能。

\item {} 
\textbf{力的平衡原理}:寻找合适的平衡元件或者借助于对称的布置,将伴随主参量生成的副参量限制在尽可能小的范围内,从而降低结构成本和消耗。

\end{itemize}
\end{quote}


\subsection{任务分配原理}
\label{unit6:id10}\begin{enumerate}
\item {} 
分功能的配置

\end{enumerate}
\begin{quote}

根据任务分配原理,每一个功能配置一个功能载体。

由于将单个任务分开后,可以对每个分功能进行合适的优化结构设计和明确的计算,因而任务分配原理具有下述优点:
\begin{itemize}
\item {} 
可以更好地对有关构件作较好的充分利用。

\item {} 
允许获得较高的工作能力。

\item {} 
保证得到正确的性能,从而支持了基本规则“明确”。

\end{itemize}

为了检验任务分配原理能否有效地应用,要对功能进行分析,并且不会在同时完成多个功能时
\begin{itemize}
\item {} 
产生约束或发生

\item {} 
彼此之间的阻碍或干扰。

\end{itemize}
\end{quote}
\begin{enumerate}
\setcounter{enumi}{1}
\item {} 
功能不同时的任务分配

\item {} 
功能相同时的任务分配

\end{enumerate}
\begin{quote}

倘若功率或量值增加到某一极值,可以将同一功能分配至多个相同的功能载体。
\end{quote}


\subsection{自助原理}
\label{unit6:id11}\begin{enumerate}
\item {} 
\textbf{概念与定义}

\end{enumerate}
\begin{quote}

按照自助原理,通过巧妙地选择系统元件以及其在系统中的布置,达到自身互相支持的效应,从而有助于较好地实现其功能。

在自助结构中,由原始作用和辅助作用组成所要求的总作用。

原始作用引入工作过程,建立必要的起始状态,并且其效果往往相当于传统的没有辅助作用的解,然而其有效程度相应地比较低。

辅助作用则是来自功能的主参量和伴随而生的副参量,只要给定它们之间的排列关系即可。
\end{quote}
\begin{enumerate}
\setcounter{enumi}{1}
\item {} 
\textbf{自加强解}

\end{enumerate}
\begin{quote}

自加强解就是在正常负荷下由作用于功能的主参量和副参量在确定的排列方式下取得某种辅助作用,从而使得总的作用加强。
\end{quote}
\begin{enumerate}
\setcounter{enumi}{2}
\item {} 
\textbf{自平衡解}

\end{enumerate}
\begin{quote}

自平衡也是在正常负荷下由作一定安置的伴生副参量相对主参量获得辅助作用,从而抵消了起始的作用,从而达到平衡,使总作用得到提高。
\end{quote}
\begin{enumerate}
\setcounter{enumi}{3}
\item {} 
\textbf{自保护解}

\end{enumerate}
\begin{quote}

在超负荷情况下,倘若没有规定的断裂状态的要求,则构件不应损坏。
\end{quote}


\subsection{稳定性和双稳定性原理}
\label{unit6:id12}\begin{enumerate}
\item {} 
稳定性原理

\end{enumerate}
\begin{quote}

在结构上设计时应预先做到干扰会引起抵消或至少自相削弱的作用。
\end{quote}
\begin{enumerate}
\setcounter{enumi}{1}
\item {} 
双稳定性原理

\end{enumerate}
\begin{quote}

在开关和保护系统中,要求具有双稳定性能。
\end{quote}


\section{结构设计准则}
\label{unit6:id13}

\subsection{概述}
\label{unit6:id14}

\subsection{考虑膨胀的合理设计}
\label{unit6:id15}
技术系统中应用的材料具有受热膨胀的性质。
\begin{enumerate}
\item {} 
膨胀现象

\item {} 
构件的膨胀

\end{enumerate}
\begin{quote}

如果温度不随时间变化,则称之为稳定膨胀。倘若温度分布随时在变化,则称之为非稳定的、亦即随时间变化的膨胀。
\end{quote}
\begin{enumerate}
\setcounter{enumi}{2}
\item {} 
构件之间的相对膨胀

\end{enumerate}
\begin{quote}

\textbf{稳定相对膨胀}
\begin{quote}

在稳定状况下,当时的平均温差与时间无关。如欲减小相对膨胀,可采取措施在线膨胀系数相同的情况下,使得温度相同,或者在不同温度下选配具有不同膨胀系数的材料。

如果材料不能够任意选择,就必须进行相应的温度协调。
\end{quote}

\textbf{非稳定相对膨胀}
\begin{quote}

若温度随时间而变化,由于各个零件的温度可能有很大的差别,以致经常产生比最后的稳定状态大很多的相对膨胀。
\end{quote}
\end{quote}


\subsection{考虑蠕变和松弛的合理设计}
\label{unit6:id16}\begin{enumerate}
\item {} 
温度下的材质性质

\end{enumerate}
\begin{quote}

设计在温度下的构件时,除了膨胀效应外,还要考虑有关材料的蠕变性质。
\end{quote}
\begin{enumerate}
\setcounter{enumi}{1}
\item {} 
蠕变

\end{enumerate}
\begin{quote}

蠕变是与所承受的应力、作用温度和时间有关的。

室温下的蠕变

极限温度以下的蠕变

极限温度以上的蠕变
\end{quote}
\begin{enumerate}
\setcounter{enumi}{2}
\item {} 
松弛

\end{enumerate}
\begin{quote}

在总伸长量不变情况下弹性伸长部分减小的过程称之为“松弛”。
\end{quote}
\begin{enumerate}
\setcounter{enumi}{3}
\item {} 
设计措施

\end{enumerate}
\begin{quote}

结构设计时,通过下列途径,使得蠕变保持在一定的允许范围内:
\begin{itemize}
\item {} 
高的弹性伸长贮备,使得由于温度变化引起的附加应力很小。

\item {} 
绝缘或构件冷却,

\item {} 
避免在非稳定过程中引起热应力的质量聚积。

\item {} 
防止材料沿着可能损害功能或者造成拆卸困难不利方向蠕变。

\end{itemize}
\end{quote}


\subsection{考虑腐蚀的合理设计}
\label{unit6:id17}
设计师必须采用适当的方案或者通过合理的结构设计防止不可容许的腐蚀现象。
\begin{enumerate}
\item {} 
腐蚀的原因和现象

\end{enumerate}
\begin{quote}

设计师采取的措施取决于腐蚀的原因和现象。
\end{quote}
\begin{enumerate}
\setcounter{enumi}{1}
\item {} 
自由表面的腐蚀

\end{enumerate}
\begin{quote}

自由表面的腐蚀可能是均匀的表面腐蚀或者是局部的有限腐蚀。

\textbf{均匀表面腐蚀}
\begin{quote}

原因:存在来自空气或介质的氧气的同时出现潮气,特别是在露点下降时更为严重。

现象:扩展性的产生均匀损耗在表面腐蚀。
\end{quote}

\textbf{凹坑腐蚀}
\begin{quote}

原因:存在具有阳性与阴性区域的腐蚀电池,主要由于材料的不均匀性、介质方面的不同浓度、或者由于区域性的不同条件,引起了腐蚀进展的不一致。
\end{quote}

\textbf{孔穴腐蚀}
\begin{quote}

原因:同坑穴腐蚀,但限制在窄小的区间里。
\end{quote}

\textbf{狭缝腐蚀}
\begin{quote}

原因:大多数是由于在缝隙中腐蚀产物的水解作用而造成电解液的酸性浓缩。
\end{quote}
\end{quote}
\begin{enumerate}
\setcounter{enumi}{2}
\item {} 
与接触有关的腐蚀

\end{enumerate}
\begin{quote}

\textbf{接触腐蚀}
\begin{quote}

原因:由于材料配对形成两种金属带有不同的电位,或者由于存在电解液,即导电的液体或潮气而使固体处于导电联结中。
\end{quote}

\textbf{沉积腐蚀}
\begin{quote}

原因:在表面或缝隙中有沉积异物,这些沉积异物在有关位置引起电位差。
\end{quote}

\textbf{相界腐蚀}
\begin{quote}

原因:由于与金属表面接触的介质从液相向气相转变,或者反之,在金属表层的突变区产生腐蚀的危险有所增加。
\end{quote}
\end{quote}
\begin{enumerate}
\setcounter{enumi}{3}
\item {} 
与应力有关的腐蚀

\end{enumerate}
\begin{quote}

\textbf{振动裂纹腐蚀}
\begin{quote}

原因:对受到交变机械应力的零件的腐蚀侵蚀会导致强度的剧烈下降。
\end{quote}

\textbf{应力裂纹腐蚀}
\begin{quote}

原因:当由于外载荷或内应力引起的静拉力和导致裂纹的特殊因素同时作用时,某些敏感材料在一定的时间后会形成穿过晶体的或晶间的裂纹。
\end{quote}

\textbf{延伸诱导腐蚀}
\begin{quote}

原因:由于反复的延伸、墩粗作用,超过极限值后,覆盖层破裂,以致不再存在自然腐蚀保护层,并且出现局部腐蚀。
\end{quote}

\textbf{流体浸蚀腐蚀、气蚀腐蚀和摩擦腐蚀}
\begin{quote}

主要的补救措施是采用流体力学或者设计的方法避免或减少浸蚀和气蚀的出现,只有做不到这点时,才考虑采用硬的表面覆盖层。
\end{quote}

\textbf{选择性腐蚀}
\begin{quote}

原因:某些组织成分或者晶界附近区域的抗腐蚀性能低于基体。
\end{quote}
\end{quote}
\begin{enumerate}
\setcounter{enumi}{4}
\item {} 
考虑腐蚀的合理设计举例

\end{enumerate}


\subsection{考虑人机工程的合理设计}
\label{unit6:id18}
人机工程从人的特性、能力和需要出发,研究人和技术产物之间的关系。

利用人机工程的知识,通过相应的结构设计达到下述要求:
\begin{itemize}
\item {} 
技术产物与人相适应,或

\item {} 
通过对人员的遴选以及教育和训练,使得人对技术活动或技术系统能合宜地适应。

\end{itemize}
\begin{enumerate}
\item {} 
人机工程基础知识

\end{enumerate}
\begin{quote}

生物力学观点

生理学观点

心理学的观点
\end{quote}
\begin{enumerate}
\setcounter{enumi}{1}
\item {} 
人的活动和人机工程条件

\end{enumerate}
\begin{quote}

在技术活动中,人可能主动地或者被动地介入到或者关联到技术活动中去。
\begin{description}
\item[{\textbf{人的主动贡献}}] \leavevmode
在技术系统中,人的作用的显明性和合理性是按效益、经济和宜人性的观点来衡量的。

\end{description}

\textbf{人的被动反应}
\end{quote}
\begin{enumerate}
\setcounter{enumi}{2}
\item {} 
对人机工程要求的认识

\end{enumerate}
\begin{quote}

对对象的考虑

对作用关系的考虑
\end{quote}


\subsection{考虑造型的合理设计}
\label{unit6:id19}\begin{enumerate}
\item {} 
任务和目的

\end{enumerate}
\begin{quote}

技术产品不应只以相应的某个功能结构从纯粹实现某个目的的意义上来满足所要求的技术功能,而且应在令人心情愉快的美学方面对人产生吸引力。
\end{quote}
\begin{enumerate}
\setcounter{enumi}{1}
\item {} 
造型合理的特征

\end{enumerate}
\begin{quote}

用所选出的技术解形成的技术功能,以及由此而产生的结构,
一般由有关零件和组件的布置和形状确定其外形构造。这样就产生了很少能改变的功能结构设计。

人们不仅感觉到这些功能性结构设计,而且感知更多确定的特征,这形成了特征结构设计。
\begin{description}
\item[{\textbf{面向市场和使用方面的特征}}] \leavevmode
对于总体结构设计基本上应达到:
\begin{itemize}
\item {} 
简单、统一、单纯、风格真实。

\item {} 
整齐、成比例、相似。

\item {} 
可描述、可解释。

\end{itemize}

\item[{\textbf{面向目的的特征}}] \leavevmode
这些特征应使得目的能够认识和可以感觉到。

\end{description}

\textbf{面向操作的特征}
\begin{description}
\item[{\textbf{面向制造厂、销售商及商标的特征}}] \leavevmode
这些特征对产品的来源、公司的风格、企业的体系进行了表达。

\end{description}
\end{quote}
\begin{enumerate}
\setcounter{enumi}{2}
\item {} 
造型的准则

\end{enumerate}
\begin{quote}

特征结构设计是通过一种特定的要求的表达来实现的。

选择一定的表达方式
\begin{itemize}
\item {} 
采用与目标相适应的明显、统一的表达方式。

\end{itemize}

构造总体外形
\begin{itemize}
\item {} 
以可标记的方式进行布置。

\item {} 
分解成界限明显的区段。

\end{itemize}

统一形状
\begin{itemize}
\item {} 
形状和位置变型少,

\item {} 
采用原则上选出的形状,规定相应的相似形状单元和相配的棱线走向。

\end{itemize}

色彩的支持
\begin{itemize}
\item {} 
色彩布置与形状布置相协调。

\item {} 
尽量采用少的色调和材料差别。

\item {} 
在采用多色彩时,规定一种特征色,并与衬色相协调。

\end{itemize}

通过图形加以补充
\begin{itemize}
\item {} 
采用风格相同的字体和图形符号。

\item {} 
通过相同的图版制造方法,

\item {} 
图形在大小、形状和色彩方面与其余部分的形状和色彩北京相协调。

\end{itemize}
\end{quote}


\subsection{考虑工艺的合理设计}
\label{unit6:id20}\begin{enumerate}
\item {} 
设计——生产的关系

\end{enumerate}
\begin{quote}

设计方面的决策对生产成本、生产时间和生产质量有重要影响。
考虑工艺的合理结构设计的目的在于:通过设计上的措施,力求生产成本和生产时间最少,
并且获得符合要求的与生产有关的质量特征。

考虑工艺的合理组合结构,通过对产品按照部件和单个零件进行分类,
以自制件或作为新零件、重复件或标准件的外购件的形式确定其工艺过程。

考虑工艺的合理工件结构,决定单个零件的加工方法,加工手段和质量。

考虑工艺的合理材料选择,从本身方面确定加工方法、加工手段、材料经济性和质量检验。

标准件和外来件的应用,对生产能力、仓库管理和经济性产生影响。

考虑工艺的合理生产文件,必须考虑生产方式、工作过程和质量检验。
\end{quote}
\begin{enumerate}
\setcounter{enumi}{1}
\item {} 
考虑工艺的合理组合结构

\end{enumerate}
\begin{quote}

根据工艺要求,可按分解、集成、联接或组合件结构方式的观点对产品组合结构进行分类。

分解结构方式,把单一个部件分解成多个在制造技术上有利的工件。
分解结构方式的优点:
\begin{itemize}
\item {} 
可以应用容易买到和得到的半成品或标准件。

\item {} 
锻件和铸件较易取得。

\item {} 
与企业生产设备相适应。

\item {} 
即使在单件和小批生产中,也提高了工件的批量。

\item {} 
减少了工件尺寸,从而易于装配和运输。

\item {} 
由于材料的均质性而比较容易保证质量的可靠性。

\item {} 
维修较容易,

\item {} 
容易适应特殊要求。

\item {} 
减少超期危险性,缩短生产流程。

\end{itemize}

缺点或使用局限性:
\begin{itemize}
\item {} 
较高的切削加工消耗。

\item {} 
较高的装配费用。

\item {} 
较高的用于保证质量的耗费。

\item {} 
由于结合部位引起功能或负荷受到限制。

\end{itemize}
\begin{description}
\item[{集成结构方式}] \leavevmode
所谓集成结构方式是把多个单个零件统一成一个工件。

\item[{联接结构方式}] \leavevmode
联接结构方式可以理解为:
\begin{itemize}
\item {} 
将多个已制成的不同胚件用不可拆方式联接在一块,成为进一步待加工的工件。

\item {} 
同时利用多种联接方法作工件的联接。

\item {} 
多种材料的组合,用以最有利地利用其性能。

\end{itemize}

\item[{组合件结构方式}] \leavevmode
如果通过分解结构形式把一个机器结构拆开,使所得到的工件和组件同样可以用于企业中其它的产品或者产品变体中,那么这就是所谓制造组合件。

\end{description}
\end{quote}
\begin{enumerate}
\setcounter{enumi}{2}
\item {} 
考虑工艺的工件合理结构设计

\end{enumerate}
\begin{quote}

设计师通过对工件形状、尺寸、表面质量、公差和配合的选择,会对下列几个方面起到影响:
\begin{itemize}
\item {} 
所考虑的制造方法。

\item {} 
可以应用的机床,包括加工和测量工具。

\item {} 
在广泛采用企业内部的重复件以及合适的标准件和外购件情况下的自制和外加工问题。

\item {} 
材料和半成品的选择及其利用和

\item {} 
质量检查的可能性。

\end{itemize}
\begin{description}
\item[{考虑原型成形工艺}] \leavevmode
对于铸造材料的构件,其结构应考虑造模合理、成形合理、铸造合理以及加工合理。
对于烧结构件,其结构应当考虑工具合理和烧结合理。

\item[{考虑变形成形工艺}] \leavevmode
对于变形成形零件的毛柸结构,应当考虑的工艺方法:自由模锻和冲模锻(压力成形)、冷挤压和拉伸(挤压成形)以及弯曲成形。

\end{description}

对于自由锻,由于没有应用复杂的锻造装置,因而只要考虑锻造合理即可。其结构设计准则为:
\begin{itemize}
\item {} 
力求形状简单,并且尽可能设计成平行的表面和采用大的圆角。

\item {} 
力求锻件不要过重,在有的情况下可以分开然后联接起来。

\item {} 
避免过大变形及过大的断面差。

\item {} 
优先采用单面冲孔或凸台。

\end{itemize}

对于模锻成形,应尽量作出工具合理,锻造合理和加工合理的结构设计。

冷挤压

拉伸

弯曲成型,应当力求其结构切断合理和弯曲合理。
\begin{description}
\item[{考虑分离工艺}] \leavevmode
结构设计准则应当是工具合理和切削合理。

\end{description}

工具合理意味着
\begin{itemize}
\item {} 
规定有足够的夹紧可能性。

\item {} 
优先采用无需重新夹持工件或工具的加工方法。

\item {} 
注意刀具必要的退刀行程。

\end{itemize}
\begin{quote}

对于所有的分离方法,切削合理意味着:
* 避免不必要的切削加工,亦即加工面、表面质量和公差限制在非要不可的范围内。
* 力求加工表面与夹持表面平行或者垂直。
* 优先采用车、钻加工方法,然后才是铣、刨加工方法。
\end{quote}
\begin{description}
\item[{考虑联接}] \leavevmode
焊接工艺过程可分为准备、焊接及后处理三个步骤。

\end{description}
\end{quote}
\begin{enumerate}
\setcounter{enumi}{3}
\item {} 
考虑工艺的材料和半成品合理选择

\end{enumerate}
\begin{quote}

按照柸件或半成品的种类、技术供货条件以及后处理和质量,根据这些特征标志选出来的材料,对下列各项产生影响:
\begin{itemize}
\item {} 
制造方法。

\item {} 
机床,包括刀具和量具。

\item {} 
材料管理性,

\item {} 
质量检查。

\item {} 
自生产和外生产问题。

\end{itemize}
\end{quote}
\begin{enumerate}
\setcounter{enumi}{4}
\item {} 
标准件和外来件的应用

\end{enumerate}
\begin{quote}

采用自制件或者外来件,取决于:
\begin{itemize}
\item {} 
件数(单件、成批或者大量生产)。

\item {} 
与订货相关的单件产品或者一个面向市场的产品系列和组合系统。

\item {} 
材料、外购件或者外协件的获取状况(成本、供货期限)。

\item {} 
企业内已有加工设备使用的可能性。

\item {} 
加工设备的配置状态。

\item {} 
已有的或争取达到的自动化程度。

\end{itemize}
\end{quote}
\begin{enumerate}
\setcounter{enumi}{5}
\item {} 
考虑工艺合理的技术文件

\end{enumerate}


\subsection{便于装配的合理设计}
\label{unit6:id21}\begin{enumerate}
\item {} 
装配操作

\end{enumerate}
\begin{quote}

装配可理解为在零件加工期间和加工之后以及在施工现场包括所有必要辅助工作的组装活动。
装配的费用和质量既决定于装配操作的种类和次数,也决定于其本身进行的情况。种类和次数则与组合结构、工件结构和生产类型有关。
\end{quote}
\begin{enumerate}
\setcounter{enumi}{1}
\item {} 
便于装配的合理组合结构

\end{enumerate}
\begin{quote}

专配合理的组合结构通过对装配操作的下述处理而达到:
\begin{itemize}
\item {} 
分解。

\item {} 
缩减。

\item {} 
统一和

\item {} 
简化。

\end{itemize}
\end{quote}
\begin{enumerate}
\setcounter{enumi}{2}
\item {} 
便于装配的结合部位合理结构设计

\end{enumerate}
\begin{quote}

通过减少、统一和简化接合部位,可以降低用于联接元件、装配操作以及对联接件质量要求的费用。
\end{quote}
\begin{enumerate}
\setcounter{enumi}{3}
\item {} 
便于装配的结合零件合理结构设计

\item {} 
应用和选择导则

\end{enumerate}
\begin{quote}

整理归纳要求表中确定及影响装配的要求和愿望:
\begin{itemize}
\item {} 
单件产品或者变型系列。

\item {} 
变型件数。

\item {} 
安全技术限制和法规限制。

\item {} 
加工条件和装配条件。

\item {} 
试验要求和质量特征标志。

\item {} 
运输要求和包装要求。

\item {} 
维护和回用方面对装配和拆卸的要求。

\item {} 
用户使用对装配操作的要求。

\end{itemize}

认真研究原理解(作用结构),特别是初步技术设计草案(组合结构),以便充分利用结构上使装配简便的可能性。
\end{quote}


\subsection{有利于标准化的合理设计}
\label{unit6:id22}\begin{enumerate}
\item {} 
标准化的目标

\end{enumerate}
\begin{quote}

标准化可理解为关于解的统一和解的确定的总概念。
\end{quote}
\begin{enumerate}
\setcounter{enumi}{1}
\item {} 
标准的种类

\item {} 
标准的准备

\item {} 
有利于标准化的合理结构设计

\end{enumerate}
\begin{quote}

标准应用的提示:
\begin{quote}

功能,规定的总功能和分功能是否能通过应用标准而得以实现?

作用原理,现有的标准是否促进合理的解的原理或者方案的进一步发展?

结构设计,在结构设计时,应注意基本标准和专业标准、特别是草图标准、设计标准、尺寸标准、材料标准和安全标准。

安全性,对于企业、工作和环境的安全性,应遵守已有的标准和法律规定。

人机工程

生产

检查,试验标准和检查规则对于保证质量很重要。

装配,正确无误的装配是通过遵守有关公差、配合和联接件的标准、以及注意试验标准和检查规则而得到保证的·。

运输

使用

维护

回用

费用,成本和时间期限可通过工厂标准而将至最低。
\end{quote}
\end{quote}
\begin{enumerate}
\setcounter{enumi}{4}
\item {} 
标准开发

\end{enumerate}


\subsection{有利于回用的合理设计}
\label{unit6:id23}\begin{enumerate}
\item {} 
概述

\end{enumerate}
\begin{quote}

为节省原材料,可考虑下列几种可能性:
\begin{itemize}
\item {} 
通过合理利用材料和减少加工废料的办法,减少材料的应用量。

\item {} 
采用代用材料,即用便宜的、可长期供应的材料代替紧缺的、因而也是价格昂贵的原材料。

\item {} 
通过取回加工废料、产品或产品零部件,将它们重新应用或利用,以实现回用。

\end{itemize}

生产废料回用是把加工废料取回,送到新的生产过程中去。

产品使用过程中的回用是在保持产品形式的情况下取回用过的产品或产品零部件,从而进入一个新的使用阶段。

旧物料回用是把用过的产品和旧物料取回送到新的生产过程中去。

在回用循环中可能有不同的回用形式,基本上可分为对产品重新应用和利用两大类。

应用的特点是继续保证产品的形状。按照产品的重新应用是否完成原有的功能,又可区分为重复应用和转换应用两类。

利用要破坏原有的产品形式,因而一开始便有较大的价值损失。按照利用时是否采用原有的制造方法,又可区分为重复利用及转换利用两类。
\end{quote}
\begin{enumerate}
\setcounter{enumi}{1}
\item {} 
回用过程

\end{enumerate}
\begin{quote}
\begin{description}
\item[{\textbf{处理}}] \leavevmode
通过加压的办法将零散的废料压实。

通过废钢剪或气剪的办法把笨重的或巨大的旧产品破裂。

利用根据锤碾原理制造的破碎装置可以将废料进行分离。

浮选装置接在粉碎装置和磨碎装置之后,用于较好地对有色金属和非金属部分进行分选。落锤机用于碎裂大的厚壁灰铸铁件。化学处理装置用于在重新冶炼前提出有害的物质和合金。

具有单一原料高回用价值率的最好废料或称高质量废料是通过在处理过程以前进行的旧产品拆卸而达到的。

设计师应通过选出的组合结构和联接工艺为经济的拆卸创造前提条件。

\item[{\textbf{整修}}] \leavevmode
为了在第一个使用阶段后重复应用和转换应用产品,要求有一个整修过程。此过程由下列步骤组成:
\begin{itemize}
\item {} 
彻底的拆卸。

\item {} 
清洗。

\item {} 
检验。

\item {} 
重复应用值得保存的部分,维修磨损区域,修整配合部分,将不能再用的零件用新的零件代替。

\item {} 
重复装配。

\item {} 
检验。

\end{itemize}

在专门的车间或产品生产部门进行整修可以以两种方式进行。第一种可能性是保持老产品的统一性不变,
即在更换和修整零件时保持其相互关系,而对其制造公差进行调整。另一种可能性则是将老产品完全拆开,
所有的零件在公差方面像新零件一样对待。

\end{description}
\end{quote}
\begin{enumerate}
\setcounter{enumi}{2}
\item {} 
设计措施

\end{enumerate}
\begin{quote}

设计过程中对回用的考虑

为了考虑回用的观点,在设计过程的各个步骤中,应注意:
\begin{quote}

在要求表中给出在考虑经济性条件下有利于处理和整修的回用条件。

在功能结构中尽量采取促进回用的功能分离。

在方案设计中优先采用促进回用的作用原理和作用结构。

特别是在确定组合结构时要注意回用的观点。
\end{quote}

在技术设计过程中考虑处理和整修的准则:
\begin{itemize}
\item {} 
采用合适的联接方式,使得拆卸和重新装配方便。

\item {} 
可能进行整修。

\item {} 
材料选择考虑利用的相容性,并且腐蚀最小。

\item {} 
便于检验和分类。

\item {} 
清洗能够彻底。

\end{itemize}

在对工艺文件进行施工设计时,也要标明回用策略和回用工艺。

\textbf{有利于处理过程的产品结构设计准则}
\begin{quote}

材料相容性:由于便于利用的单独材料产品很少出现,应当尽量采用在利用时相容、并且从而能经济地、高质量地利用的不可分整体性的材料组合。

材料分离:倘若产品的不可分的零件和组件不能达到相容的话,则应通过附加的结合部位使之能进一步分开,以便于在处理的过程中通过拆卸而将不相容的材料分离。

便于处理的结合部位:能够用于优质的和经济的处理的结合部位应当是易于拆卸、人手可及的、并且应尽量安排在产品的外部。

贵重材料:贵重、紧缺的材料应当特别仔细地按照拆开的要求进行布置,并且作上标记。

危险材料:对于在处理或直接利用时会对人、设备和环境产生危险的材料,无论如何要可拆分地、也就是可拆出地进行布置。
\end{quote}

\textbf{有利于整修过程的产品结构设计准则}
\end{quote}
\begin{enumerate}
\setcounter{enumi}{3}
\item {} 
有利于回用的合理结构设计举例

\end{enumerate}
\begin{quote}

滑动轴承座的旧材料回用

家用机械的旧材料回用
\end{quote}


\section{克服设计错误、干扰量影响和冒险}
\label{unit6:id24}

\subsection{错误和干扰因素的识别}
\label{unit6:id25}
是否能及早地识别错误性质,原则上正确地探知干扰的影响,并且在必要的情况下减少这种影响。
\begin{enumerate}
\item {} 
错误树分析

\end{enumerate}
\begin{quote}

错误性质和干扰影响可以应用与方法学进程有关的所谓错误树分析方法有效地查找出来。

由方案设计阶段一斤得知包含各个可以实现的分功能的功能结果。通过技术设计的加工,
同样也知道了所要求的从功能。用它对功能结构进行补充。对于某一部件或者要检查的区域,
所有必要的功能都可以描述出来。
\end{quote}
\begin{enumerate}
\setcounter{enumi}{1}
\item {} 
干扰量影响

\end{enumerate}
\begin{quote}

当分功能的排列和组合不明确时,干扰来自功能结构。当物理效应不能在高度和均匀性上取得假设的效果时,
则干扰主要来自作用原理。所选择的理论结构由于不稳定的材料性质以及由加工和装配引起的形状、
位置和表面的偏差,就会得到与原先规定不同的特性。
\end{quote}
\begin{enumerate}
\setcounter{enumi}{2}
\item {} 
处置方法

\end{enumerate}
\begin{quote}

为了寻找和排除错误和干扰因素,可以采取下述方法:
\begin{itemize}
\item {} 
功能识别和否定。

\item {} 
按照为技术设计过程制定的导则,寻找不能实现功能的原因。

\item {} 
确定出现错误状态所必须具备的那些先决条件。

\item {} 
在设计范围内通过其它的解、改善的解或者在制造、装配、运输、使用和维护中的检验措施,引入相应的补救办法。

\end{itemize}
\end{quote}


\subsection{考虑冒险的合理结构设计}
\label{unit6:id26}\begin{enumerate}
\item {} 
冒险的对策

\end{enumerate}
\begin{quote}

在考虑冒险的合理设计中,技术上与经济上的冒险应当协调,一方面保证制造厂在经验方面获得有用的利益,
另一方面也保证用户获得可靠的、无损害的运行状态。
\end{quote}
\begin{enumerate}
\setcounter{enumi}{1}
\item {} 
考虑冒险的合理结构设计举例

\end{enumerate}


\section{技术设计的评价}
\label{unit6:id27}
在技术设计阶段,特别是当评价仅仅是对最终有效的技术设计方案进行判断时,
则评价也同时意味着重要的对薄弱部位的寻找。


\section{方案设计及技术设计举例}
\label{unit6:id28}

\chapter{第7章 系列产品和组合产品的开发}
\label{unit7::doc}\label{unit7:id1}

\section{系列产品}
\label{unit7:id2}
对制造厂来说,产品系列化有如下优点:
\begin{itemize}
\item {} 
在很多应用场合中,设计工作按一定的配置原则只需进行一次。

\item {} 
只重复进行一定批量的生产,因此较为经济。

\item {} 
能够获得高质量。

\end{itemize}

对于用户来说,它有下列优点:
\begin{itemize}
\item {} 
产品价廉物美。

\item {} 
供货快捷。

\item {} 
备件的获得和补充不成问题。

\end{itemize}

但对制造厂家和用户,产品系列化的缺点:
\begin{itemize}
\item {} 
参数选择受到限制,不一定能得到最佳的工作特性。

\end{itemize}

所谓系列产品,就是这样的技术产物(机器、部件或零件),在广泛的应用范围内:
\begin{itemize}
\item {} 
在有多个参数等级的情况下

\item {} 
用尽可能相同的制造方法

\item {} 
以相同的解

\item {} 
实现相同的功能

\end{itemize}

如果除了参数分等级外还要附带满足其它功能,则在开发系列产品同时,还要开发组合产品系统。

一个系列产品开发的实质,是从要开发的系列产品(机器、部件或零件)中的一种结构参数格式出发,
按照一定的规律推导出其它各级大小的结构参数规格。其中那个原始规格的设计方案称为基础技术设计草案,
由此推演出的其它参数规格称为后续技术设计草案。


\subsection{相似定律}
\label{unit7:id3}
将“模型”看作原始的技术设计草案,即“基础技术设计草案”,而模型的最终“设计”看作是系列中的某个规格,
即“后续技术设计草案”。

与模型技术相比,系列产品开发通常还有其它目的,要:
\begin{itemize}
\item {} 
尽可能用相同的材料及

\item {} 
相同的工艺

\item {} 
达到同样高的材料利用率。

\end{itemize}

当基础技术设计草案与后续技术设计草案中,至少有一个物理量之比保持为常量即不变时,便可以说具有相似性。

如果基本参数比中,常数超过一个,则具有专门的相似性,并能对其做出特殊描述。
当长度比和时间比同时为不变量时,称之为运动相似。当长度比和力比为常数时,则为静力相似。

等应力的相似性


\subsection{十进制几何标准数系}
\label{unit7:id4}
应用标准数系的优点:
\begin{itemize}
\item {} 
适应现有的需求,这时各段可用不同的分级,以使参数分级与需求的重点相吻合。

\item {} 
由于采用了基于标准数系的规格,从而减少了尺寸不同的方案数目,从而节省了在制造过程中用于样板、

\end{itemize}

夹具和量具上的费用。
* 因为数系的各项积和商仍然是一个几何数系的项,所以主要是乘和除的参数选择与计算就变得容易。
* 如果一个构件或一台机器的尺寸是一个几何级数的项,在作线性方法或缩小时,
如果放大或缩小的系数同样取自数系,则所得尺寸的数字亦在该数系中。
* 一些重要的参数级独立地增大,而这些参数级与其它现有的或将要采用的数系是相互协调的。


\subsection{参数分级的选择}
\label{unit7:id5}
参数分级是根据多方面的观点来确定的:
\begin{itemize}
\item {} 
其一是从市场情况出发。市场通常希望有较细的分级,以便用尽可能适合的机器和设备参数来满足顾客的要求。

\item {} 
第二个观点是来自设计和制造方面。参数分级必须从技术与经济观点来选择,

\end{itemize}

一方面分级要足够细满足技术上的要求,但另一方面分级又要足够粗,以便使规格类型较少及各种零件的批量大,
从而做到系列的经济加工。

要确定最佳的参数分级,首先要掌握有关状况和需求方面的、有说服力和必须是可靠的信息:
\begin{itemize}
\item {} 
关于各个结构参数的市场(销售)需求预测。

\item {} 
在规格调整而出现缺档时的市场特点。

\item {} 
不同参数的分级的加工成本及加工工时,首先要掌握变化着的加工总成本,以及

\item {} 
不同的参数分级下的产品特性。

\end{itemize}

参数范围可按不变的或变化的级比来划分,可以在标准数系之间跳跃和在较粗和较细分的标准数系之间跳动。

在分级时,必须把自变参数与因变参数区分开来。


\subsection{几何相似的产品系列}
\label{unit7:id6}

\subsection{半相似产品系列}
\label{unit7:id7}
由于下列原因,会迫使系列明显偏离几何相似而要求其按别的规律增长:
\begin{itemize}
\item {} 
需要优先满足的相似定律。

\item {} 
需要有点满足的任务书要求及

\item {} 
需要优先满足的加工经济性要求。

\end{itemize}

在这些情况下就导致开发所谓半相似系列。
\begin{enumerate}
\item {} 
需要优先满足的相似定律
\begin{quote}

重力的影响

热力过程的影响

不同的相似关系
\end{quote}

\item {} 
需要优先满足的任务要求

\end{enumerate}

这种情况往往是由于人机关系而产生的。人在工作中接触的所有构件,
都必须符合人的生理条件和人体尺寸。它们一般都不能随系列产品的名义参数而改变。

一个要优先满足的任务要求,也可能是出于纯粹技术条件上的原因,这时外购件和生产产品并无几何相似的尺寸。
\begin{enumerate}
\setcounter{enumi}{2}
\item {} 
需要优先满足的加工经济性要求

\end{enumerate}

如果一个产品系列本身必须较细分级,则可以对其零件和部件采用较粗分级的方法,以求获得较大的工件批量,
从而可能达到比较经济的加工。

如果这些零部件所在不同的范围容许,功能上当然也没问题,就可以在细分级的产品系列中对它们进行粗分级。
那么,这些部分与其紧接着的部分,便成了半相似系列。
\begin{enumerate}
\setcounter{enumi}{3}
\item {} 
利用指数方程进行调整

\end{enumerate}

所谓指数方程是一种简单的辅助手段,它根据相似关系的种类来考虑上面三点所述的条件,
并用它们来开发半相似产品系列。
\begin{enumerate}
\setcounter{enumi}{4}
\item {} 
例子

\end{enumerate}


\subsection{产品系列的开发}
\label{unit7:id8}
产品系列开发过程概括如下:
\begin{enumerate}
\item {} 
制定基础技术设计草案,它是从一个打算开发的产品系列中产生,或者从已有的产品定出。

\item {} 
根据相似定律确定物理的函数关系(指数)。

\item {} 
确定参数分级和参数线图的应用范围。

\item {} 
调整由理论所得的系列,使之与优化标准或工艺条件相适应,并将这些偏差表示在参数线图上。

\item {} 
通过对各部件或极端规格的临界区按比例画出结构图,以检查该产品系列。

\item {} 
改进和完善文件资料,这些资料是在确定系列和制订加工文件的过程中所需要的。

\end{enumerate}


\section{组合产品}
\label{unit7:id9}
如果一个产品目录中有一个或多个参数规格必须满足各种不同的功能,便要分别设计为数众多的不同产品,
这意味着要付出相当大的设计和制造费用。合理的做法是,用确定的零部件(功能结构块)组合成各种要求的功能变型。
一个这样的组合可以利用组合产品原理来实现。

组合产品指的是这样一些机器和零部件,它们
\begin{itemize}
\item {} 
往往作为具有不同解的结构块通过相互组合

\item {} 
来实现不同的总功能。

\end{itemize}


\subsection{组合产品系统学}
\label{unit7:id10}
组合产品系统由组合块构成,这些结构块可拆或不可拆地拼合成一体。

要区分功能结构块和制造结构块。功能结构块是从满足技术功能的观点来确定的,
因此它能够自身或通过其它功能结构块组合来实现技术功能。制造结构块则不是根据其功能,
而纯粹是按照制造技术观点加以确定的。
\begin{itemize}
\item {} 
基本功能在一个系统中是基本的、反复出现和不可缺少的。它们基本上是不变的。

\item {} 
辅助功能用于联接和接通,它通过辅助结构块来实现,这些辅助结构块通常为联接元件和接头。

\item {} 
特殊功能是特殊的、补充的和任务书特别要求的分功能,它不一定必须在各种总功能变型中反复出现。特殊功能由特殊功能结构块来实现,特殊功能结构块表现为对基本结构块的一种特殊补充或作为一个附件,因而是可能结构块。

\item {} 
适应功能是为了适应其它系统和边界条件所必须的。它通过适应结构块在物质上起作用。

\end{itemize}

在组合系统中可能会出现难以预见的为任务书特别要求的功能,这些功能通过非结构块来实现。
用了它,就成了由结构块和非结构块联合而成的一个混合系统。

一个结构块的含义可以理解为一个组合产品系统内的一种等级排列。

用模块使一个产品分段具体化:
\begin{itemize}
\item {} 
在单件生产且对功率和效率的要求常有很大变化的情况下,将组合产品划分为结构段(模块)是十分适当的。

\item {} 
为了划分组合产品界限,可用带有有限个可以预见的变体数目的组合目录,在所谓闭式系统内,

\end{itemize}

对组合产品系统的范围和可能性加以说明。


\subsection{组合产品的开发进程}
\label{unit7:id11}
\textbf{阐明任务书}

在阐述要求和愿望时,必须借助于诸如任务所示的祝特征,仔细而全面地拟订为产品目录所实现的各个任务。
一个产品系统要求表的特点是要求表有好几个总功能。因此就产生了该组合产品系统要满足的总功能变体。

\textbf{建立功能结构}

有了功能结构,即将所需的总功能划分为分功能,就已相当大程度地确定了系统的组合结构。

建立功能结构时强调下列目标:
\begin{itemize}
\item {} 
力求用尽可能少和容易实现的基本功能的组合,来实现所要求的总功能。

\item {} 
总功能按要求划分为若干基本功能,必要时还划分出辅助功能、特殊功能和适应功能,划分时应这样进行,

\end{itemize}

即需求数量大的变体主要由基本功能组合,需求量很少的变体附带地作为特殊功能和适应功能。
* 将若干个分功能集中到一个结构块上。

\textbf{寻找作用原理和解的变体}

首先要找到这样一些原理,它们允许在保持相同的作用原理及基本相同的结构设计的情况下,产生各种变体。

\textbf{选择与评价}

如果在已进行的工作步骤中,已找到若干个解的方案,就应按技术和经济准则对其评价,从而选出最有利的原理解。

\textbf{总技术设计方案的制订}

在结构设计指导原则的前提下,必须力求对组合产品所要求的基本结构块、特殊结构快、
辅助结构块和适应结构块这样进行结构设计,即应使相同的和反复出现的工件数量大,
而且这些工件用尽可能少的毛柸和加工工序制成。

最佳的分解度,受到很多标准的影响:
\begin{itemize}
\item {} 
在注意到误差传播后果的同时,必须满足要求及质量指标。

\item {} 
总功能变体应通过结构块(零件和部件)的简单装配产生。

\item {} 
结构块只分解到功能能力所要求的、质量所要求的和成本所允许的程度。

\item {} 
当组合产品系统被用户当作总系统,并由用户自己通过不同结构块的组合编排而成各种组合产品系统变体时,

\end{itemize}

那些常用的结构块特别需要在强度和磨损方面,按要求的尽可能相同的工作寿命或易更换性,进行分解和安排。
* 在根据成本和制造时间确定分解度时,要从整个组合产品进行考虑。

\textbf{技术文件的制订}

技术文件应能在任务完成过程中简单地、并且尽可能地用数据处理方法辅助编制,此外,
它还能对预期的总功能变体作进一步处理。

绘制相应图纸时,恰当的产品编号和分类很重要,因为这位结构块(零件和部件)相互成链奠定了基础。

各结构块与产品变体的关系,在零件明细表中确定下来。采用所谓变体零件表制订零件表是很合适的,因为变体零件表是在产品组合结构的基础上制定的,并突出了必须结构块和可能结构块。

平行编码特别适合于组合产品系统图纸和零件表编号,平行编码包含着一个与零件明确和不可更换的标记相一致的识别号码,以及一个按功能编排和用来调用这些零部件的分类号码。


\subsection{组合产品系统的优点和局限}
\label{unit7:id12}

\subsection{例子}
\label{unit7:id13}

\chapter{第8章 施工设计}
\label{unit8::doc}\label{unit8:id1}

\section{施工设计的步骤}
\label{unit8:id2}
“施工设计”就是要对技术产物的组合结构加以补充,亦即对其形状、尺寸和各零件的表面状态、所选用的材料、
生产和使用的可能性的检验及最终的成本等等作出最后的规定,并完成物质上得以实现和利用的有约束力的图纸和其他文件。

施工设计的重点是拟定加工文件,特别是绘制零件图或加工图、部件的装配图、以及必要的总图和零件表。

最终技术设计方案的详细化不仅要绘出零件,同时还要就它的形状、材料、
表面质量和公差配合等方面的细节进行优化。优化的目的是获得高利用率,
并得到一个既便于生产又经济合算的详细结构设计。

完善文件,可以通过编制加工规程、装配规程、运输规程以及使用和操作说明书来进行。

技术文件的检查,对后续生产过程有重要意义。特别是对零件图和零件表进行下列方面的检查尤为重要:
\begin{itemize}
\item {} 
对标准特别是对厂标准的遵守情况,

\item {} 
明确的和便于加工的尺寸标注,

\item {} 
其他必要的制造说明,

\item {} 
是否采用外购件。

\end{itemize}


\section{技术文件系统学}
\label{unit8:id3}

\subsection{制品分解}
\label{unit8:id4}
制造文件结构化或者编排的基础是所谓制品分解,它反映在设计部门制订的成套图纸和零件表中。
所谓制品分解,就是将制品分为较小的单位。

制品分解可能导致得出依功能而定的结构或依生产而定和依装配而定的结构。


\subsection{图纸系统}
\label{unit8:id5}
技术图纸按表达种类分为:
\begin{itemize}
\item {} 
草图。不一定受形式和规则的约束,多数是徒手和按粗略的比例画出。

\item {} 
尽可能按比例绘制的图纸。

\item {} 
作简化表达的尺寸图。

\item {} 
布置图。例如位置图。

\item {} 
为了便于理解的线图和示意图等。

\end{itemize}

在方案设计阶段最重要的是草图和示意图,因为它们用于帮助求解,并且是提供信息的手段。
用粗略比例和按比例画的图纸,用作技术设计阶段进行结构构思和计算的工作基础和传递信息的辅助手段,以及在施工设计阶段结束后作为技术文件。

技术图纸按制备种类分为:
\begin{itemize}
\item {} 
原始图,它们是复制的基础。

\item {} 
预印图纸,往往不按比例。

\end{itemize}

关于技术图纸内容,在衡量时看一个产品在一张图中的完整程度,这里区分为:
\begin{itemize}
\item {} 
总图,

\item {} 
部件图,

\item {} 
零件图,

\item {} 
毛柸图,

\item {} 
布置图,

\item {} 
模型图,和

\item {} 
示意图,

\end{itemize}

成套图纸是为了一个目的而汇编起来的所有图纸资料的总和。

与图纸内容密切相关的是绘制目的,由此出发,技术图纸可以分为:
\begin{itemize}
\item {} 
结构方案图。

\item {} 
生产图。

\end{itemize}

成产图可以进一步分为:
\begin{itemize}
\item {} 
完善程度不同的加工图。

\item {} 
组装图。

\item {} 
备件图。

\item {} 
检验图。

\item {} 
安装图和

\item {} 
发货图。

\end{itemize}

根据便于加工和便于装配的图纸分类,成套图纸基本上首先由下列部分组成:
\begin{itemize}
\item {} 
制品的总图(主图)。

\item {} 
不同层次(综合性)的好几个部件图。它表示出若干个零件如何装配成一个生产和安装单元。

\item {} 
零件图。对应于不同的加工工序,它还有所分类。

\end{itemize}

集合图用于同类零件的不同规格,并且集合同属的零件和简单的部件作为组图,
以及可将一个构件的不同品种或规格集中在一种图上作为品种图。


\subsection{零件表系统}
\label{unit8:id6}
所有属于一个制品的零件表的总和称为成套零件表。

零件表种类说明了制品分解和生产级如何反映到零件表的结构上来的。

零件表种类可以定义如下:
\begin{itemize}
\item {} 
数量概况零件表,这种表只包含制品的零件及它的物品号码和数量说明,多处用到的零件表在表中只出现一次,

\end{itemize}

但制品的全部零件编号均予以列入。
* 结构零件表重现了包括所有部件和零件的制品结构,表中每个部件直接分解到最高等级。
组件和零件的分解通常与生产流程相对应。
* 用变体零件表的概念来表征零件表的特殊形式,大部分组件或零件相同的不同制品或部件都安置在这种表中。

为了使不同制品和重复部件的零件表内容不经修改即能使用,可将总零件表按结构块种类划分为各个独立的部分。
基于这个目的,零件表的形式如下:
\begin{itemize}
\item {} 
组合产品零件表包括同属的组件或零件,而不首先涉及确定的制品。

\end{itemize}

按使用目的零件表还可以采用其他不同格式:
\begin{itemize}
\item {} 
设计零件表,在这类表中设计人员按功能观点组合零件和建立相应的制品结构。

\item {} 
所谓加工零件表和装配零件表,是根据加工和装配的观点划分零件表的结构和内容的。

\item {} 
个别情况下,还有备用表,核算表和备件表。

\end{itemize}

将与零件有关的信息归为零件基本数据,而将属于产品特征结构的信息,分开归为制品结构数据。
这两种数据的内容分别为:
\begin{itemize}
\item {} 
基本数据例如用于识别零件的图号或物品号、材料说明、数量单位和零件种类;

\item {} 
结构数据例如部件编号和位置编号,修改注明,订货号和特定的关键码。

\end{itemize}

零件表逆向使用的形式,称为零件使用说明。


\section{对象物的标记}
\label{unit8:id7}

\subsection{编号技术}
\label{unit8:id8}
对号码系统通常的要求是:
\begin{itemize}
\item {} 
鉴别,即根据特征标志能明确和不可互换地识别一个对象物。

\item {} 
分类,即根据确定的概念,能够编排物品和物品的特性。

\item {} 
识别和分类应能分别处理。

\item {} 
号码系统从其结构起都应允许作较大的扩展。

\item {} 
必须保证提取时间短,那怕是人工处理,管理必须简单。

\item {} 
必须与数据处理技术的要求相协调。

\item {} 
用逻辑系统结构,甚至使企业外的人都容易理解,力求专业术语明确和具有强的标记能力。

\item {} 
号码系统应具有用于处理和输出各种信息而且便于设计的结构,它由设计人员建立,也为设计人员所用,

\end{itemize}

特别用于图纸和零件表编号。
* 一个对象物的号码应该是不变的,并且与这个对象物装在何种产品上和它是自制件还是外购件无关。

在选择或确定合适的号码系统时,必须注意企业的情况和目标设置。其中重要的影响因素有:
\begin{itemize}
\item {} 
产品目录的种类和复杂性;

\item {} 
生产批量;

\item {} 
用户服务机构,备件和经销机构;

\item {} 
机构情况;

\item {} 
编号的目的。

\end{itemize}
\begin{enumerate}
\item {} 
物品号码系统

\end{enumerate}
\begin{quote}

物号编码是对一种物品的统一号码。

物品号码系统的结构,可以编成平行号码系统和复合号码系统。

在平行号码系统中,为一个识别号码加编一个或多个与识别无关的分类号码。这样的平行编码的优点,
在于有较大的柔性和扩充的可能性,因为两个分系统实际上是互不相关的。

在复合号码系统中总号码由分类和识别号码两部分组成,这两部分之间以固定的形式相联,计数号码依附于分类号码部分。
\end{quote}
\begin{enumerate}
\setcounter{enumi}{1}
\item {} 
分类系统

\end{enumerate}
\begin{quote}

通常进行粗分类和细分类,粗分类一般广泛注意区分下列物品范围:
\begin{itemize}
\item {} 
技术的、经济的和组织的文件;

\item {} 
原材料、半成品等;

\item {} 
外购件,即不是自己设计和生产的对象物;

\item {} 
自己设计的零件;

\item {} 
自己设计的部件;

\item {} 
制品,产品;

\item {} 
辅助原料和生产原料;

\item {} 
夹具,工具;和

\item {} 
加工设备。

\end{itemize}

分了除了使企业内部在完成任务过程中信息交换合理外,
其重要性还在于设计人员能迅速和广泛地取得已经设计或作为库存件的现成相同件或相似件的信息。
寻找重复件的系统的效能有多高,取决于分类系统及其等级和分类的特征标志内容,
以及信息的输入尤其是输出的种类。
\end{quote}


\subsection{物品特征标志}
\label{unit8:id9}
用物品特征标志或其变化形式来标征对象物,它与周围的情况无关。


\chapter{第9章 成本估算}
\label{unit9::doc}\label{unit9:id1}

\section{可影响的成本}
\label{unit9:id2}
为了降低成本,在尽可能早的设计阶段中就开始进行考虑成本合理性的优化工作。
因为通过选择一个有利的原理解,比通过纯粹加工方面的措施,产品的成本一般说来能有较大的降低。
另一方面,往往只有重要的结构性的改变才能在制造阶段引起较高的成本变化。

一个产品在制造时的总成本,按其结算方式分为单件成本和公共成本。
单件成本是指可以直接划归为一个成本承担者的费用。还有是不可能直接归于此类的,这些费用被称为公共成本。

成本的确定还和订单数量、工作强度以及批量大小有关。材料费用、制造工资费用、辅助材料和消耗材料的费用,
都将随销售量的增长而有所提高。这些成本将作为可变成本引入到成本估算中去。
而不变成本则是指那些在一定的时期内不变的费用。

制造成本是与产品制造有关的、用于材料、制造并包括其附属特殊费用的总成本。


\section{陈本估算的方法}
\label{unit9:id3}

\subsection{基本原理}
\label{unit9:id4}

\subsection{相对成本比较}
\label{unit9:id5}

\subsection{材料成本部分的估算}
\label{unit9:id6}

\subsection{应用回归分析的估算}
\label{unit9:id7}

\subsection{应用相似关系的推算}
\label{unit9:id8}\begin{enumerate}
\item {} 
以基础技术设计草案为基础

\item {} 
以工艺元素为基础

\item {} 
以回归分析为基础

\end{enumerate}


\subsection{成本结构}
\label{unit9:id9}

\section{价值分析}
\label{unit9:id10}
价值分析的目的是降低成本。其中有两个工作要点:首先,为了对所给的解进行讨论和评价,
所属企业范围之内应该保证一种跨学科的合作,即在销售、采购、设计、
制造和核算各方面的专家之间的联系应予保证。另一个工作要点是,为了实现总功能而需作功能分解工作,
包括分解为复杂程度较低的分功能,以及进一步分配给功能载体即部件和零件的分解工作。


\section{降低成本的规律}
\label{unit9:id11}
一些通用原则:
\begin{itemize}
\item {} 
力求低的复杂程度,亦即少的零件数和制造工序数。

\item {} 
宜用小的结构尺寸,以求较低的材料成本,因为材料成本随着尺寸,通常是直径,而超线性地增长。

\item {} 
为了减少一次性费用部分,尽可能采用较多的零件数量。

\item {} 
提出有限度的精度要求,亦即应该允许有尽可能大的公差及粗糙度。

\end{itemize}


\chapter{第10章 计算机辅助设计}
\label{unit10::doc}\label{unit10:id1}

\section{数据处理}
\label{unit10:id2}

\subsection{数据技术的基础知识}
\label{unit10:id3}
数据处理装置、通用的数据处理系统或者设计用CAD系统,能够接收、存贮、加工和输出数据。

数据按种类可分为字符型和图像型。字符型数据主要用于计算机操作和文件加工,
图像型数据则主要用于工作结果的图形表达和几何对象的描述。

为了接收、存贮、加工和输出数据,必须进行合适的数据管理。

借助于一个与应用程序严格分开的数据管理系统,可以实现将必要的和待加工的数据,
特别是以计算机内部模型的型式存在的数据,放在各个数据文件或者一个数据库中存贮和管理。

数据库和数据库管理系统组成数据库系统。


\subsection{对象的计算机内部描述}
\label{unit10:id4}
为了在计算机内部产生一个真实的技术对象,首先必须定义一个实质模型。

从实质模型所含的对象出发,可以通过格式化完成信息模型的描述。

体模型有实体结构和表面结构两种。所谓实体结构,就是在事实上用一些实体进行描述。
这些实体CAD系统中也被称之为基本实体。在具有表面结构的体模型中,
则采用由轮廓元素和点所组成的一系列表面来描述对象。

信息模型的图形是用计算机内部模型和计算机内部描述来表示的,
它通过所谓生成模型(联接模型,生产模型)来实现。


\subsection{CAD系统的结构}
\label{unit10:cad}
设计部门所使用的程序系统基本上应该由下列各部分程序合成:
\begin{itemize}
\item {} 
通信部分,负责从设计师处输入数据和向设计师输出数据;

\item {} 
方法部分,包括与专业有关的工作模块,并利用这些模块实现模型化、提供信息和进行计算;

\item {} 
数据管理部分(数据库管理系统),它组织方法、算法和通信部分与数据库系统或数据文件之间的全部数据传输和数据存贮;

\item {} 
数据库,包括全部存贮的几何数据和非几何数据,这些数据是设计方法以及设计师与CAD系统之间通信所必需的。

\end{itemize}


\subsection{计算机装备和运行}
\label{unit10:id5}

\section{设计各阶段中计算机的使用}
\label{unit10:id6}

\subsection{概论}
\label{unit10:id7}

\subsection{经过选择的例子}
\label{unit10:id8}\begin{enumerate}
\item {} 
\textbf{计算}

\end{enumerate}
\begin{quote}

计算操作在设计中是进行校核(对一个已经设计好的零件的特性计算)、
参数选择(按给定要求确定一个零件的尺寸)和优化(通过参数变异追求最优解)所必须的。

\textbf{校核程序}

\textbf{参数选择计算程序}
\begin{description}
\item[{\textbf{优化程序}}] \leavevmode
可以看作是较高级的参数选择程序。它与参数选择程序的区别在于,它力求通过被考察参数的变异,使一个特定值或一个特定函数值趋向极值(最优值)。

\item[{\textbf{仿真程序}}] \leavevmode
要研究和表示对象的主要特征相对于时间的关系,例如运动过程,为这一目的服务的程序,称为仿真程序。

\end{description}
\end{quote}
\begin{enumerate}
\setcounter{enumi}{1}
\item {} 
\textbf{结构设计}

\item {} 
\textbf{提供信息}

\item {} 
\textbf{CAD的其它方面应用}

\end{enumerate}


\section{使用CAD系统时的工作技术}
\label{unit10:id9}
表示任务的模型,可以是标准化了的任务要求的特征标志。利用这些特征标志,
可以把提出的任务编成程序,输入计算机。

表示功能的模型,可以是一般功能或者逻辑功能的框图(功能结构)。

表示结构的模型


\subsection{一个几何模型的产生}
\label{unit10:id10}
用于产生一个几何模型的主要工作技术,就是以基本实体为基础和以表面为基础的处理方法。

\textbf{以基础实体为基础的处理方法}
\begin{quote}

利用这种处理方法可以由一系列几何单元实体出发,建立一个三维的构件几何形体。
\end{quote}

\textbf{以表面为基础的处理方法}
\begin{quote}

这一工作技术最适宜于辅助技术设计,因为它:
\begin{itemize}
\item {} 
能够使空间边界条件看得清楚,

\item {} 
能够用预先确定的几何元素使图形表示结构化,

\item {} 
能够利用辅助几何图形的元素产生棱边。

\end{itemize}
\end{quote}

\textbf{宏技术}
\begin{quote}

宏技术的基础是应用已经存在的零部件的几何模型。
\end{quote}

\textbf{改建技术}
\begin{quote}

这一工作技术的基础是,从手工输入的二维视图出发,进一步自动化地构成三维模型。
\end{quote}

\textbf{输入方法}


\subsection{一个几何模型的改变}
\label{unit10:id11}

\subsection{技术设计时的工作技术}
\label{unit10:id12}
\textbf{要减少剖面图和视图的数量}

\textbf{通过输入基本体和面使几何图形的产生得到简化}

\textbf{有许多决定必须很早就作出}

\textbf{先进行粗略结构设计,然后进行局部的精确结构设计}

\textbf{简单几何图形的生成促使产品结构趋向统一和更便于加工制造}

\textbf{巨大的图形生成工作量迫使设计师利用重复零件的标准件}


\subsection{例子}
\label{unit10:id13}

\section{CAD技术的可能性和局限性}
\label{unit10:id14}

\section{CAD的引进}
\label{unit10:id15}
\textbf{教育水平}

\textbf{可接受性}

\textbf{经济性}


\section{软件设计}
\label{unit10:id16}

\subsection{工作步骤}
\label{unit10:id17}\begin{description}
\item[{\textbf{阐明并精确规定任务书}}] \leavevmode
这一阶段的结果是一个具有约束力的要求表和一个评价系统。要求表的内容主要是描述一个初步的、
用户所要求的功能结构。

\item[{\textbf{方案设计阶段}}] \leavevmode
建立功能结构和数据结构:功能结构由许多分功能组成。这些分功能之间用数据流相连,
并且随后依靠程序技术通过功能模块加以实现。

在功能结构化的同时,必须建立数据结构。一个数据结构由分数据区域组成。
这些分数据区域合在一起构成了程序系统的总数据存贮量。

为已经定义的分功能寻找作用原理:寻找作用原理以实现某个分功能时,
要注意到作用结构的特征。这里可以把作用原理分成下列几种:
\begin{itemize}
\item {} 
结构原理:数据模块从数据库中取出和重新存入的算法。

\item {} 
操作原理:通过对输入数据进行数学和逻辑操作产生输出数据。

\item {} 
通信原理:把计算机内部数据提供给用户,或对用户数据左前处理,使其适合于计算机内部处理。

\end{itemize}

作用原理的组合:在复杂程度相同的水平上,可以按照功能结构实现作用原理的组合。

原理解决方案的非格式具体化:找到的总解决方案应构成文件,文件的完善程度以是否能够根据它进行评价为准。

\item[{\textbf{技术设计阶段}}] \leavevmode
程序的技术设计阶段与机械系统的技术设计阶段不同之处主要在于:
程序的精细结构设计的全部工作都在制定技术文件阶段中才得以完成。

把系统设计方案结构化,亦即分成确定结构的主模块,其它主模块和副模块:
\begin{itemize}
\item {} 
主模块:为实现主要的应用功能所必须的功能模块或数据模块,

\item {} 
副模块:用以实现不很重要的应用功能的功能模块或数据模块。

\end{itemize}

因此这类功能在要求表中可能是以愿望的方式提出。

确定结构的主模块的粗略结构设计:首先进行主模块的粗略结构设计,
亦即先借助于数据纸把它们具体化到必要的程度,接着进行主功能模块的粗略结构设计,
与此同时要注意到数据模块说明。

\end{description}

\textbf{粗略技术设草案的选择}

\textbf{其余主模块的粗略结构设计}

\textbf{副模块的粗略结构设计}

\textbf{按技术经济标准进行评价}
\begin{description}
\item[{\textbf{制定技术文件阶段}}] \leavevmode
在注意到副模块的前提下进行主模块的精细结构设计:
精细结构设计在这里可以理解为对每一功能模块和数据模块都必须执行下列步骤:
\begin{itemize}
\item {} 
通过加上为所使用的程序语言特有的细节,而使数据和功能模块的说明得以实现。

\item {} 
研究和确定与语言有关的信息,研究和确定实现输入输出的方式。

\item {} 
把功能模块的粗略结构设计方案具体化,使之称为框图结构。

\item {} 
使数据模块具体化,并实现数据模块说明。

\item {} 
把框图结构和数据模块说明转换成所选择的程序语言。

\item {} 
编译已经程序化的功能模块,必要时让各部分分别试运转,消除错误。

\item {} 
就是否必须作超越模块界限的修改进行检查。

\item {} 
制定技术文件。

\end{itemize}

副模块的精细结构设计:

把设计好的模块综合成为一个能够运行的试验版本:现在可以把已经设计好的、编译过的、
而且已经试算过的各个模块,合成为一个可以运行的程序系统,进行总试验。

\end{description}

\textbf{根据技术经济标准进行评价}

\textbf{正确性分析和可靠性分析}
\begin{description}
\item[{\textbf{薄弱环节分析和效率分析}}] \leavevmode
完善技术文件:最终有效的程序文件至少应该包括:
\begin{itemize}
\item {} 
要求表,

\item {} 
评价系统,

\item {} 
功能结构及其说明,

\item {} 
功能等级,

\item {} 
数据结构及其说明,

\item {} 
寻找和选择解决原理的技术文件,

\item {} 
具有粗略技术设计草案、框图结构或数据说明的模块说明。

\item {} 
在各个阶段进行评价和选择决策的文件,

\item {} 
带有对最终程序产品的注解的源代码,

\item {} 
使用手册,

\item {} 
带有试验数据说明的试验记录,

\item {} 
关于正确性、可靠性、效率和薄弱环节分析的文件。

\end{itemize}

\end{description}


\subsection{结构设计建议}
\label{unit10:id18}
明确的结构设计意味着:每一个功能模块精确地完成它在功能结构中被赋予的任务。

简单的结构设计意味着:所有算法应该尽可能用线性化描述。

安全的结构设计意味着:

直接和短的控制流原理意味着:每一个模块的程序流程应该尽可能短而且没有回跳的途径。

效率协调原理是指在程序中一起工作的模块或者分程序的运行时间应该差不多相同。

任务分配原理是通过把复杂的程序分解成许多具有基本结构特征的模块来实现的。

在结构设计中注意到可能的失误,这意味着程序模块都具有定义了的接口,使其更易于更换。

按标准化原则进行结构设计意味着应用标准程序、标准数据结构和标准控制结构,此外还可以应用技术文件标准。

便于制造的结构设计:能运行的程序的产生和装机都要靠源程序的支持。


\chapter{Indices and tables}
\label{index:indices-and-tables}\begin{itemize}
\item {} 
\emph{genindex}

\item {} 
\emph{modindex}

\item {} 
\emph{search}

\end{itemize}



\renewcommand{\indexname}{Index}
\printindex
\end{document}
